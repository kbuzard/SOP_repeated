%C:\Users\Kristy\Dropbox\Research\SOP\SOP_repeated\JIE_RR
\documentclass[12pt]{article}

\addtolength{\textwidth}{1.4in}
\addtolength{\oddsidemargin}{-.7in} %left margin
\addtolength{\evensidemargin}{-.7in}
\setlength{\textheight}{8.5in}
\setlength{\topmargin}{0.0in}
\setlength{\headsep}{0.0in}
\setlength{\headheight}{0.0in}
\setlength{\footskip}{.5in}
\renewcommand{\baselinestretch}{1.0}
\setlength{\parindent}{0pt}
\linespread{1.1}

\usepackage[pdftex,
bookmarks=true,
bookmarksnumbered=false,
pdfview=fitH,
bookmarksopen=true]{hyperref}

\usepackage{amssymb, amsmath, amsthm, bm}
\usepackage{graphicx,csquotes,verbatim}
\usepackage[backend=biber,block=space,style=authoryear]{biblatex}
\setlength{\bibitemsep}{\baselineskip}
\usepackage[american]{babel}
%dell laptop
\addbibresource{C:/Users/Kristy/Dropbox/Research/xBibs/tradeagreements.bib}
%\addbibresource{C:/Users/Kristy/Documents/Dropbox/Research/xBibs/tradeagreements.bib}
\renewcommand{\newunitpunct}{,}
\renewbibmacro{in:}{}


\DeclareMathOperator*{\argmax}{arg\,max}
\usepackage{xcolor}
\hbadness=10000

\newcommand{\ve}{\varepsilon}
\newcommand{\ov}{\overline}
\newcommand{\un}{\underline}
\newcommand{\ta}{\theta}
\newcommand{\al}{\alpha}
\newcommand{\Ta}{\Theta}
\newcommand{\expect}{\mathbb{E}}
\newcommand{\Bt}{B(\bm{\tau^a})}
\newcommand{\bta}{\bm{\tau^a}}
\newcommand{\btn}{\bm{\tau^n}}
\newcommand{\btw}{\bm{\tau^{tw}}}
\newcommand{\ga}{\gamma}
\newcommand{\Ga}{\Gamma}
\newcommand{\de}{\delta}

\newtheorem{result}{Result}

\begin{document}
\begin{center}
Notes on 2nd round of revisions for JIE of SOP$\_$Repeated
\end{center}

\vskip.3in
\section{Dixit, Grossman and Helpman analogy}
DGH97 paper:
\begin{itemize}
	\item Has to be a truthful contribution schedule
		\begin{itemize}
			\item Proposition 3: $G(a^0,e^T(a^0,u^0)) = \max_a G(a,0)$
			\item $e^T(a^0,u^0)$ is essentially $\phi(a^0,u^0)$, which is defined implicitly in eqn3 (p. 760) as
				\[
				  U\left[ a, \phi(a^0,u^0)\right] = u^0
				\]
					\begin{itemize}
						\item Careful: truthful contribution schedule is not a best response function. But the lobby will still have to be best responding in equilibrium.
						\item Truthful contribution schedule is a device for solving equilibrium in their model. Doesn't mean I have to follow it (I think; I hope)
					\end{itemize}
			\item if $G = W + g(e)$ and $e=0$ and $g(0) = 0$, then $a^* = \tau^{\text{opt}}$
			\item Then rewrite as 
				\[
				  G(\tau^0,e^T(\tau^0,u^0)) = \max_a G(\tau^{\text{opt}},0)
				\]
					\begin{itemize}
						\item Right hand side provides a number
						\item Ignoring arguments of $e^T$ function, LHS traces out $\left(\tau,e\right)$ pairs that satisfy the equation given $g(e)$.
						\item For lobby to be best responding, it MUST pick the pair that maximizes $\pi(\tau) - e$. \textit{This} concern must be what sets $u^0$.
					\end{itemize}
		\end{itemize}
	\item Corollary to Prop 1 / Prop 3: Gov't gets utility equal to outside option. Is this true when there is just one lobby?
	\item Combining the two previous facts (if true in my case), then it must be that gov't getting outside option will set $u^0$ (eqm utility) and anchor contribution schedule (just have to be careful of zero contributions)
\end{itemize}

\vskip.2in
What editor proposes:
\begin{itemize}
	\item Lobby offers contribution schedule (can be very simple: just one $(e,\tau)$ pair, $e=0$ for everything else)
	\item Government maximizes $W + g(e)$
		\begin{itemize}
			\item Note that $CS_X + \ga(e) \cdot PS_X + CS_Y + PS_Y +TR = W + \left( \ga(e) - 1 \right) PS_X$
		\end{itemize}
	\item Sectioning from my paper:
		\begin{itemize}
			\item[3.1] Same (execs)
			\item[3.2] Trade war
				\begin{itemize}
					\item Government chooses unilateral $\tau$ as $\frac{\partial W}{\partial \tau} + \frac{\partial g(e)}{\partial \tau} = 0$
						\begin{itemize}
							\item How to think about lobby's contribution schedule?
							\item DGH Proposition 1: principal (lobby) has to provide at least agent's (gov'ts) outside option; as long as this constraint is satisfied, lobby can propose $\tau$ and payment that maximizes his own utility
							\item DGH Proposition 3 (p. 760-61): simplifies so we don't need to look for contribution functions, only eqm values
						\end{itemize}
				\end{itemize}
		\end{itemize}
\end{itemize}

\vskip.2in
Comparisons
\begin{itemize}
	\item Note that in GH94, claim is that IN EQUILIBRIUM, government behaves as if it maximizes a weighted sum of the groups' utilities ($1+a$ for those represented by lobbies; $a$ for those not represented; pg. 10 of PDF). But this is in equilibrium: there's a lot going on out of equilibrium that leads us there!
		\begin{itemize}
			\item I haven't proved this to myself, but that must also depend on lobbies having all the bargaining power
		\end{itemize}
	\item In DGH, $G(\tau^0,e^0) = \max_\tau (\tau,0)$
		\begin{itemize}
			\item This is \un{not} generally true in my model
			\item I think the slippage is in bargaining power. DGH paradigm assumes lobby essentially make TIOLI offer.
			\item $\ga(e)$ formulation in essence distributes bargaining power more generally
				\begin{itemize}
					\item Is it okay to characterize it this way?
					\item If so, does bargaining power vary with effort? Seems mixed up with diminishing returns to effort.
				\end{itemize}
			\item $W + \Phi(e)$ of Limao and Tovar gives decreasing returns; then explicitly models bargaining power through Nash bargain instead of menu auction (as far as I can tell--I can't find it clearly specified).
		\end{itemize}
	\item In DGH, essentially lobby chooses its favorite $(\tau,e)$ pair from among all the possible ones that make the Gov't indifferent.
		\begin{itemize}
			\item Is it possible that this is equivalent to $\frac{\partial \pi}{\partial e} = 1$?
			\item seems like the pair that satisfies that equation might not be available
			\item When I calculate the various $(\tau,e)$ pairs to 'compare' government welfare levels, this comes from the government welfare function: each $e$ induces an optimal $\tau$ and then I get an optimal value function
				\begin{itemize}
					\item Lobby doesn't care about these varying welfare levels. Chooses $(\tau,e)$ pair that maximizes $\pi - e$
				\end{itemize}
		\end{itemize}
\end{itemize}

\vskip.2in
Examples
\begin{itemize}
	\item Use numeric example from BS2005 in R (DGH.r)
	\item Isomorphism I've already calculated: if $G = W + e$ then $\ga(e) = 1 + \frac{e}{\pi_x}$
	\item Lobby's optimization:
		\[
		  W(\tau,e(\tau)) = W(0,0)
		\]
		\[
		  W(\tau) + e = W(0)
		\]
		\[
		  e = W(0) - W(\tau)
		\]
			
			\begin{itemize}
				\item This forms a contribution schedule
				\item Lobby chooses the pair $(\tau,e)$ that maximizes $\pi(\tau) - e$
			\end{itemize}
	\item Make sure these are unilateral changes in $\tau$ in the program
	\item Then compare to my old way: $\frac{\partial \pi}{\partial \tau} \frac{\partial \tau}{\partial \ga} \frac{\partial \ga}{\partial e} = 1$ ($e$ changes with $\tau$ directly depending on shape of $\ga$, then government maximizes w.r.t. $\tau$
		\begin{itemize}
			\item NEXT: go into R program and sort it out
		\end{itemize}
\end{itemize}

\end{document}