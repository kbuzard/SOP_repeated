\documentclass[12pt]{article}

\addtolength{\textwidth}{1.4in}
\addtolength{\oddsidemargin}{-.7in} %left margin
\addtolength{\evensidemargin}{-.7in}
\setlength{\textheight}{8.5in}
\setlength{\topmargin}{0.0in}
\setlength{\headsep}{0.0in}
\setlength{\headheight}{0.0in}
\setlength{\footskip}{.5in}
\renewcommand{\baselinestretch}{1.0}
\setlength{\parindent}{0pt}
\linespread{1.1}

\usepackage{amssymb, amsmath, amsthm, bm}
\usepackage{graphicx,csquotes,verbatim}
\usepackage[backend=biber,block=space,style=authoryear]{biblatex}
\setlength{\bibitemsep}{\baselineskip}
\usepackage[american]{babel}
%dell laptop
\addbibresource{C:/Users/Kristy/Dropbox/Research/xBibs/tradeagreements.bib}
%\addbibresource{C:/Users/Kristy/Documents/Dropbox/Research/xBibs/tradeagreements.bib}
\renewcommand{\newunitpunct}{,}
\renewbibmacro{in:}{}


\DeclareMathOperator*{\argmax}{arg\,max}
\usepackage{xcolor}
\hbadness=10000

\newcommand{\ve}{\varepsilon}
\newcommand{\ov}{\overline}
\newcommand{\un}{\underline}
\newcommand{\ta}{\theta}
\newcommand{\al}{\alpha}
\newcommand{\Ta}{\Theta}
\newcommand{\expect}{\mathbb{E}}
\newcommand{\Bt}{B(\bm{\tau^a})}
\newcommand{\bta}{\bm{\tau^a}}
\newcommand{\btn}{\bm{\tau^n}}
\newcommand{\ga}{\gamma}
\newcommand{\Ga}{\Gamma}
\newcommand{\de}{\delta}

\begin{document}
\begin{center}
Literature on Asymmetric Punishments
\end{center}


New section on asymmetric punishments (addresses, in part, Giovanni's $\#$7)
\begin{itemize}
	\item Constrain to $T$-period class, now asymmetric---punish deviator more
	\item There is literature on this
		\begin{itemize}
			\item Bown 2002/2004
				\begin{itemize}
					\item Shocks, no enforcement
					\item Distinction is between legal and illegal
						\begin{itemize}
							\item Legal: TOT terms cancel, local effect only
							\item Illegal: TOT effect + local effect
						\end{itemize}
				\end{itemize}
			\item Beshkar 2010 EER (a)
				\begin{itemize}
					\item truthful revelation, no transfers, one shot
					\item asymmetry in size of punishment: compensation award through tariffs doesn't have to be as big as initial harm in order to induce truthful revelation
				\end{itemize}
			\item Beshkar 2010 JIE (b)
				\begin{itemize}
					\item GATT's instantaneous reciprocity rule (p. 39): immediate suspension of concessions under GATT escape clause
					\item 4 years with no retaliation under WTO Agreement on Safeguards
					\item No asymmetry as far as I can see on quick skim
				\end{itemize}
			\item Martin and Vergote
				\begin{itemize}
					\item Private info (random shock to import-competing sectors), no transfers, repeated game
					\item Retaliation is the cost for overstating one's own value of $\ga$--it's what delivers incentive compatibility
					\item Distinction is between reciprocity and retaliation
						\begin{itemize}
							\item Reciprocity: higher contemporary tariff (they have pre-play communication in a mechanism design framework)
							\item Retaliation: higher tariffs in the future
						\end{itemize}
					\item In contrast to Riezman 1991 (who has symmetric equilibria \`{a} la Green and Porter and then necessarily lower welfare), they have the same welfare level in punishment, just redistributed across players
						\begin{itemize}
							\item My note: this has flavor of the two different types of renegotiation issues: move inside the frontier or along it
						\end{itemize}
					\item They have an optimal level of asymmetry
					\item They show that asymmetry is necessary to deliver the efficient outcome (FLM Folk Theorem)
				\end{itemize}
			\item Hungerford 1991
				\begin{itemize}
					\item Asymmetric shocks, TOT is signal, NTBs
					\item Retaliation by going to optimal tariff level
					\item Retaliatory period: one country retaliates for past defection, defecting country plays BR (defect before other country can react)
					\item GATT (p. 364)
						\begin{itemize}
							\item Add cost of retaliation, but fast so retaliation starts in period $t+1$ for trigger in period $t$
							\item Investigation is domestic (hence cost), but must ``detect'' NTB in order to punish
							\item Countries don't spend enough on investigation to discourage NTBs entirely because they can't always detect them
						\end{itemize}
				\end{itemize}
			\item Riezman 1991
				\begin{itemize}
					\item Symmetric punishments
				\end{itemize}
			\item Bagwell (2008): commensurate vs. disproportionate retaliation
				\begin{itemize}
					\item disproportionate retaliation can compensate trading partner, who otherwise loses trade volume
					\item here, degree of disproportion increases in size of original violation: has to compensate for larger trade volume loss (p.15 of pdf)
				\end{itemize}
			\item Bagwell 2009: shocks are persistent
			\item Furusawa 1999: larger $\delta$ not always beneficial (fn 46, p. 42 of BBS lit review)
				\begin{itemize}
					\item Context: comparing the $\de_i$ of two different countries, which one is in a stronger negotiating position
					\item Repeated game, first stage of Rubinstein bargaining, then repeated interaction to support agreed-upon tariffs
						\begin{itemize}
							\item Rubinstein bargaining / asymmetric Nash bargaining that is its limit captures effect of difference in players' patience levels on their bargaining power and therefore the bargaining outcome
							\item ``On one hand, patience pays in the negotiation phase since it enhances bargaining power. On the other hand,
impatience pays in the implementation phase, since the resulting greater incentive to defect enables a government to claim a larger share of the fruits of cooperation. We find that the former effect outweighs the latter if the time lag between a defection and punishment in the implementation phase is short, and vice versa.''
							\item ``Kovenock and Thursby (1992) incorporate in their model that the GATT dispute settlement procedures may delay punishment. The US super 301, on the other hand, can be viewed as shortening the response lag''
							\item ``We also show in Section 5 that the basic model can be applied to the case in which a deviation from cooperation is detected with uncertainty, by merely replacing the response lag in the implementation phase with the expected delay to a detection. We find that an improvement of monitoring technology enhances the relative bargaining position of the relatively more patient country.''
						\end{itemize}
				\end{itemize}
			\item Ludema 2001
				\begin{itemize}
					\item I like the way he states the agreement / punishment path (p. 362)
					\item Nash tariffs for punishment with DSP; autarky for no DSP, so not asymmetric
				\end{itemize}
			\item Cotter and Mitchell 1997
				\begin{itemize}
					\item Focuses on renegotiation-proofness
					\item Countries initially negotiate a treaty that specifies particular agreement tariff levels and particular punishment tariff levels to be used if either country violates the agreement
					\item The one-period Nash equilibrium will not support a renegotiation-proof tariff agreement. Although the simple punishment strategies used here are more general than those used in most earlier papers, they are still not completely general, since, for example, the duration and severity of the punishment do not vary with the seriousness of the violation.
					\item We have been forced to simplify some of the institutional features of international trade agreements. First, we ignore lags in punishment. It is possible for countries to violate the trade agreement and avoid punishment within the WTO as long as they repent before the punishment begins (Mitchell, 1997). As a result, the modeling of punishments has been purposefully general enough to cover all types of punishment, including those not sanctioned by the WTO. We also simplify the dispute settlement procedures by specifying particular punishments for each country ahead of time, as part of a trade treaty.
					\item We expect that punishments will not harm the punishing country, unless that is unavoidable, in which case they will be as short as possible
					\item simple punishment strategy: three different tariff combinations are specified in advance: one to use during regular periods, one to use when punishing home, and one to use when punishing foreign. Punishment phases last a fixed number of periods: $T_h$ for Home, $T_f$ for Foreign.
						\begin{itemize}
							\item They also have a nice description of the strategies on p. 353-354
						\end{itemize}
					\item There is a lot more here, but it's mixed up with renegotiation; big thing that I can see easily is that the punishments \textit{can} be asymmetric in the tariff levels as well as the length
				\end{itemize}
		\end{itemize}
	\end{itemize}	
\end{document}