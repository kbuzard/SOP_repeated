\documentclass[12pt]{article}

\addtolength{\textwidth}{1.4in}
\addtolength{\oddsidemargin}{-.7in} %left margin
\addtolength{\evensidemargin}{-.7in}
\setlength{\textheight}{8.5in}
\setlength{\topmargin}{0.0in}
\setlength{\headsep}{0.0in}
\setlength{\headheight}{0.0in}
\setlength{\footskip}{.5in}
\renewcommand{\baselinestretch}{1.0}
\setlength{\parindent}{0pt}
\linespread{1.1}

\usepackage{amssymb, amsmath, amsthm, bm}
\usepackage{graphicx,csquotes,verbatim}
\usepackage[backend=biber,block=space,style=authoryear]{biblatex}
\setlength{\bibitemsep}{\baselineskip}
\usepackage[american]{babel}
%dell laptop
\addbibresource{C:/Users/Kristy/Dropbox/Research/xBibs/tradeagreements.bib}
%\addbibresource{C:/Users/Kristy/Documents/Dropbox/Research/xBibs/tradeagreements.bib}
\renewcommand{\newunitpunct}{,}
\renewbibmacro{in:}{}


\DeclareMathOperator*{\argmax}{arg\,max}
\usepackage{xcolor}
\hbadness=10000

\newcommand{\ve}{\varepsilon}
\newcommand{\ov}{\overline}
\newcommand{\un}{\underline}
\newcommand{\ta}{\theta}
\newcommand{\al}{\alpha}
\newcommand{\Ta}{\Theta}
\newcommand{\expect}{\mathbb{E}}
\newcommand{\Bt}{B(\bm{\tau^a})}
\newcommand{\bta}{\bm{\tau^a}}
\newcommand{\btn}{\bm{\tau^n}}
\newcommand{\ga}{\gamma}
\newcommand{\Ga}{\Gamma}
\newcommand{\de}{\delta}

\begin{document}
\begin{center}
JIE R$\&$R of SOP$\_$Repeated
\end{center}



\begin{itemize}
	\item Take out renegotiation
		\begin{itemize}
			\item Add more basic tradeoff
			\item (??) Draw inverted U for lobby
			\item Now my short punishments don't rest on renegotiation
				\begin{itemize}
					\item So now, for main analysis, must assume that we're constraining attention to a certain class of punishments: symmetric, and ``Punish for $T$ periods then go back to cooperation''
						\begin{itemize}
							\item Go back to start if deviate should work for governments, but I think I need something else for lobbies since they would like that
						\end{itemize}
					\item Can I show that mine are optimal in this class?
					\item Will look at asymmetric punishments in later section
				\end{itemize}
		\end{itemize}
	
	\item New section on asymmetric punishments (addresses, in part, Giovanni's $\#$7)
		\begin{itemize}
			\item Constrain to $T$-period class, now asymmetric---punish deviator more
			\item There is literature on this (see new-lit.tex)
				\begin{itemize}
					\item Bown 2002/2004: I don't think there's any reason to constrain to reciprocal ``legal'' punishments
					\item Martin and Vergote: timing. But I don't think their contemporaneous is realistic. They have the same welfare level in punishment, just redistributed across players
					\item Hungerford 1991: one country retaliates for past defection (?)
					\item Bagwell (2008): commensurate vs. disproportionate retaliation
						\begin{itemize}
							\item disproportionate retaliation can compensate trading partner, who otherwise loses trade volume
							\item here, degree of disproportion increases in size of original violation: has to compensate for larger trade volume loss (p.15 of pdf)
						\end{itemize}
					\item Cotter and Mitchell (1997): different punishments for each country
				\end{itemize}
			\item In this setting, can you achieve lower $\tau^A$ with asymmetric?
			\item Have to check lobby conditions
				\begin{itemize}
					\item Do they change over the course of the punishment?
						\begin{itemize}
							\item Joel thinks they'll be tightest at beginning of punishment phase
						\end{itemize}
					\item How asymmetric can they get?
						\begin{itemize}
							\item Is it hard to make punishment really asymmetric b/c of presence of lobby?
							\item If so, this puts some constraint on asymmetry of punishment
						\end{itemize}
				\end{itemize}
		\end{itemize}
	
	\item $\#2$ is not what I thought it was
		\begin{itemize}
			\item Giovanni's concern: when determining $\ov{e}$, I need to take into account that $\tau^b$ depends on $\ov{e}$
				\begin{itemize}
					\item I'm almost certain that I do this, but I'm also sure now that I don't explain it at all in the text
					\item It is true that the severity of the punishment for deviating does not depend on $\tau^b$, and that this means that $\tau^b$ will maximize current payoff (actually, continuation value?). So clearly $\tau^b$ is a function of $e^b$
				\end{itemize}
			\item Sweep through to make sure all analysis takes account of this concern
				\begin{itemize}
					\item Jan. 17: Decided I need to hold off until I reformulate since I don't have much in the text; I'm going to have to add more.
				\end{itemize}
			\item Maybe need to change notation on $\tau^B(\ga(e))$ to be clear
			\item Need to explain mechanics of $\ov{e}(\tau^B)$ relationship MUCH better
				\begin{itemize}
					\item End of first para. of section 3.2
					\item Paragraph following equation eq:lobtw; also above this paragraph (top of pg. 9)
						\begin{itemize}
							\item Technical part of 3rd condition needs to be re-stated
							\item Whole passage needs to be re-stated. At least some of the conditions are just what needs to be true for something to be a break tariff. What of this am I assuming? What can I show is a result of lobbying/legislature behavior?
						\end{itemize}
					\item Possibly just before start of section sec:structure
				\end{itemize}
		\end{itemize}
	
	\item email Giovanni
		\begin{itemize}
			\item How to satisfy an author who thinks the results are not ``particularly interesting or surprising'' and has not given a clear indication of what it is he wants
			\item Should I try going to linear supply/demand system?
			\item thank him
			\item ``I want to be very clear that I understand that my previous discussion did not make clear [sic]''
			\item ``I just want to know if this is along the right lines''
		\end{itemize}
\end{itemize}

\vskip1in
Smaller points
\begin{itemize}
	\item Reviewer 1, $\#2$ goes away with renegotiation
	\item Need thorough lit review of finite punishments
		\begin{itemize}
			\item Green Porter in game theory
			\item Is there anything in trade?
		\end{itemize}
	\item fix discounting to $\frac{1 - \de^{T+1}}{1-de}$ and $\frac{\de - \de^{T+2}}{1-de}$ in draft
\end{itemize}


\vskip1in
Midwest, April 2015
\begin{itemize}
	\item James Lake: what about collusion between the lobby and legislatures?
	\item Maurizio: ``trade war'' should be to MFN
\end{itemize}

\end{document}