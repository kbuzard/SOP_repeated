\documentclass[12pt]{article}

\addtolength{\textwidth}{1.4in}
\addtolength{\oddsidemargin}{-.7in} %left margin
\addtolength{\evensidemargin}{-.7in}
\setlength{\textheight}{8.5in}
\setlength{\topmargin}{0.0in}
\setlength{\headsep}{0.0in}
\setlength{\headheight}{0.0in}
\setlength{\footskip}{.5in}
\renewcommand{\baselinestretch}{1.0}
\setlength{\parindent}{0pt}
\linespread{1.1}

\usepackage{amssymb, amsmath, amsthm, bm}
\usepackage{graphicx,csquotes,verbatim}
\usepackage[backend=biber,block=space,style=authoryear]{biblatex}
\setlength{\bibitemsep}{\baselineskip}
\usepackage[american]{babel}
%dell laptop
\addbibresource{C:/Users/Kristy/Dropbox/Research/xBibs/tradeagreements.bib}
%\addbibresource{C:/Users/Kristy/Documents/Dropbox/Research/xBibs/tradeagreements.bib}
\renewcommand{\newunitpunct}{,}
\renewbibmacro{in:}{}


\DeclareMathOperator*{\argmax}{arg\,max}
\usepackage{xcolor}
\hbadness=10000

\newcommand{\ve}{\varepsilon}
\newcommand{\ov}{\overline}
\newcommand{\un}{\underline}
\newcommand{\ta}{\theta}
\newcommand{\al}{\alpha}
\newcommand{\Ta}{\Theta}
\newcommand{\expect}{\mathbb{E}}
\newcommand{\Bt}{B(\bm{\tau^a})}
\newcommand{\bta}{\bm{\tau^a}}
\newcommand{\btn}{\bm{\tau^n}}
\newcommand{\ga}{\gamma}
\newcommand{\Ga}{\Gamma}
\newcommand{\de}{\delta}

\begin{document}
\begin{center}
JIE R$\&$R of SOP$\_$Repeated
\end{center}

\vskip.3in
Notes from phone call with Joel, 9/17
\begin{itemize}
	\item Change language throughout to make clear it's a subgame-perfect Nash-reversion within a period
	\item Be more careful about how foreign responds (already taken care of in repeated game support for main construction--done on 9/17)
	\item For main nash-reversion construction: by definition, continuation payoffs don't depend on what we do today
		\begin{itemize}
			\item During punishment sequence, don't condition on what happens from period to period
		\end{itemize}
	\item In asymmetric case, can I make punishments such that players don't condition on lobbying effort across periods? Would need to condition everything on leg's response (higher tariff than it's supposed to choose)
\end{itemize}



\newpage
Intro re-write
\begin{itemize}	
	\item Now my short punishments don't rest on renegotiation
		\begin{itemize}
			\item So now, for main analysis, must assume that we're constraining attention to a certain class of punishments: symmetric, and ``Punish for $T$ periods then go back to cooperation''
			\item Can I show that mine are optimal in this class?
		\end{itemize}
	
	\item New section on asymmetric punishments (addresses, in part, Giovanni's $\#$7)
		\begin{itemize}
			\item Constrain to $T$-period class, now asymmetric---punish deviator more
			\item There is literature on this (see new-lit.tex)
				\begin{itemize}
					\item Bown 2002/2004: I don't think there's any reason to constrain to reciprocal ``legal'' punishments
					\item Martin and Vergote: timing. But I don't think their contemporaneous is realistic. They have the same welfare level in punishment, just redistributed across players
					\item Hungerford 1991: one country retaliates for past defection (?)
					\item Bagwell (2008): commensurate vs. disproportionate retaliation
						\begin{itemize}
							\item disproportionate retaliation can compensate trading partner, who otherwise loses trade volume
							\item here, degree of disproportion increases in size of original violation: has to compensate for larger trade volume loss (p.15 of pdf)
						\end{itemize}
					\item Cotter and Mitchell (1997): different punishments for each country
				\end{itemize}
			\item In this setting, you can achieve lower $\bta$ with asymmetric
		\end{itemize}
	
	\item Editor point 2: when determining $\ov{e}$, I need to take into account that $\tau^b$ depends on $\ov{e}$
		\begin{itemize}
			\item It is true that the severity of the punishment for deviating does not depend on $\tau^b$, and that this means that $\tau^b$ will maximize current payoff (actually, continuation value?). So clearly $\tau^b$ is a function of $e^b$
			\item Need to explain mechanics of $\ov{e}(\tau^B)$ relationship MUCH better
				\begin{itemize}
					\item Make Corollary 1 into Lemma 1 just before Result 1; add new Lemma 2; old Lemmas 1 and 2 renumbered to 3 and 4
					\item Improved discussion in sections 2-4
				\end{itemize}
		\end{itemize}
	
	\item Note on providing intuition for when trade agreements can be made (i.e. when an interior solution exists from section 4)
\end{itemize}

\vskip.5in
For letter		
\begin{itemize}
	\item thank him
	\item ``I want to be very clear that I understand that my previous discussion did not make clear [sic]''
\end{itemize}


\vskip1in
Smaller points
\begin{itemize}
	\item Reviewer 1, $\#2$ goes away with renegotiation
	\item Need thorough lit review of finite punishments
		\begin{itemize}
			\item Green Porter in game theory
			\item Is there anything in trade?
		\end{itemize}
	\item fix discounting to $\frac{1 - \de^{T}}{1-\de}$ (period after punishment is discounted $\de^T$) and $\frac{\de - \de^{T+1}}{1-\de}$ (period after punishment is discounted $\de^{T+1}$) in draft
	\item Clarify assumption that legislature's unilateral optimization problem has unique solution
\end{itemize}


\vskip1in
Midwest, April 2015
\begin{itemize}
	\item James Lake: what about collusion between the lobby and legislatures?
	\item Maurizio: ``trade war'' should be to MFN
\end{itemize}


\end{document}