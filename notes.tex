%C:\Users\Kristy\Dropbox\Research\SOP\SOP_repeated\JIE_RR
\documentclass[12pt]{article}

\addtolength{\textwidth}{1.4in}
\addtolength{\oddsidemargin}{-.7in} %left margin
\addtolength{\evensidemargin}{-.7in}
\setlength{\textheight}{8.5in}
\setlength{\topmargin}{0.0in}
\setlength{\headsep}{0.0in}
\setlength{\headheight}{0.0in}
\setlength{\footskip}{.5in}
\renewcommand{\baselinestretch}{1.0}
\setlength{\parindent}{0pt}
\linespread{1.1}

\usepackage[pdftex,
bookmarks=true,
bookmarksnumbered=false,
pdfview=fitH,
bookmarksopen=true]{hyperref}

\usepackage{amssymb, amsmath, amsthm, bm}
\usepackage{graphicx,csquotes,verbatim}
\usepackage[backend=biber,block=space,style=authoryear]{biblatex}
\setlength{\bibitemsep}{\baselineskip}
\usepackage[american]{babel}
%dell laptop
\addbibresource{C:/Users/Kristy/Dropbox/Research/xBibs/tradeagreements.bib}
%\addbibresource{C:/Users/Kristy/Documents/Dropbox/Research/xBibs/tradeagreements.bib}
\renewcommand{\newunitpunct}{,}
\renewbibmacro{in:}{}


\DeclareMathOperator*{\argmax}{arg\,max}
\usepackage{xcolor}
\hbadness=10000

\newcommand{\ve}{\varepsilon}
\newcommand{\ov}{\overline}
\newcommand{\un}{\underline}
\newcommand{\ta}{\theta}
\newcommand{\al}{\alpha}
\newcommand{\Ta}{\Theta}
\newcommand{\expect}{\mathbb{E}}
\newcommand{\Bt}{B(\bm{\tau^a})}
\newcommand{\bta}{\bm{\tau^a}}
\newcommand{\btn}{\bm{\tau^n}}
\newcommand{\btw}{\bm{\tau^{tw}}}
\newcommand{\ga}{\gamma}
\newcommand{\Ga}{\Gamma}
\newcommand{\de}{\delta}

\newtheorem{result}{Result}

\begin{document}
\begin{center}
Miscellaneous notes on JIE R$\&$R of SOP$\_$Repeated
\end{center}

\vskip.3in
\section{Legislative constraint}
Legislative constraint as a function of $e$
\begin{itemize}
	\item I thought it would be positive at $e=0$ and turn negative as $e$ increases
	\item What does it mean that for some values it's negative at 0, becomes positive, and then goes negative again?
		\begin{itemize}
			\item For sure I have to be careful in numerical examples
		\end{itemize}
\end{itemize}

\newpage
\section{Numerical examples}

$\de_L = \de_{ML} =.95$ \\
		\begin{tabular}{crrrrr}
			      & E=.35 & E=.4 						& E=.41						& E=.42						& E=.45\\
			$\tau^{tw}$	&		&									&									& .074						& .0654\\		
			$e^{tw}$  	&		&									&									&     						& .00123\\		
			T = 2 &       & .07500 					&									&									& .057407\\
			T = 3 &       & .074716 				& .070243					& .066284					& \textbf{.0570802}\\
			T = 4 &       & \textbf{.074708}& \textbf{.070233}& \textbf{.066275}& .0570806\\
			T = 5 &       & .074795 				& .07033 					& .06638					& .057185\\
			T = 6 & .1080 & .07492 & &&\\
			T = 7 & .1081 & 			 					& 								&									& .057\\
			T = 8 & .10814 & 			 & & &
		\end{tabular}
	
\vskip.5in
I have another sheet of notes that conflicts with the first column. It just says ``$\de=.95$'':
		\begin{tabular}{cr}
			      & E=.35 \\
			$\tau^{tw}$	&	.1213	\\		
			$e^{tw}$  	&	.006003	\\		
			T = 3 & .1023044      \\
			T = 4 & \textbf{.1022411} \\
			T = 5 & .1022427      \\
			T = 6 & .10227 \\
			T = 7 & .102305 \\
		\end{tabular}
	
\vskip.5in
This one just says ``$\de=.99$'':\\
		\begin{tabular}{crr}
			      & E=.4 & .39\\
			$\tau^{tw}$	&	&	\\		
			$e^{tw}$  	&	&	\\		
			T = 2 & .068510 & .06391      \\
			T = 3 & \textbf{.068261} & \textbf{.06374}      \\
			T = 4 & .068289 & .06378      \\
			T = 5 & .06841 & .06391      
		\end{tabular}

\vskip.5in
This has the note, ``This at least works in the direction I thought it would'' with ``$\de_L=.94, \ \de_{ML}=.95$'':\\
		\begin{tabular}{cr}
			      & E=.4 \\
			$\tau^{tw}$		&	\\		
			$e^{tw}$  		&	\\		
			T = 4 & .07464 \\
			T = 5 & \textbf{.07421} \\
			T = 6 & .07481      \\
			T = 7 & .07492       
		\end{tabular} \\
(``Really want to know if reducing $\de_L$ --- making future term less important --- will give me the $\sigma$ result I've been after; really, no result at all; depends on other parameters.)

\vskip.5in
Some summaries
\begin{itemize}
	\item $E=.4$, $\de_L=.99, \ \de_{ML}=.95$, $e_{tw} = .00232$, $\tau^{tw} = .08185$. Optimal $\tau^a = .07494$ at $T = 3$.
	\item $E=.5$, assume I kept $\de_L=.99, \ \de_{ML}=.95$. Optimal $\tau^a = .04864$ at $T = 3$.
	\item $E=.4$, $\de_L=\de_{ML}=.99$, $e_{tw} = .00232$, $\tau^{tw} = .08185$. Optimal $\tau^a = .07470$ at $T = 3$.
	\item $E=.4$, $\de_L=.99, \ \de_{ML}=.5$, $e_{tw} = .00232$, $\tau^{tw} = .08185$. Optimal $\tau^a = .07802$ at $T = 2$.
	\item $E=.4$, $\de_L=.99, \ \de_{ML}=.75$, $e_{tw} = .00232$, $\tau^{tw} = .08185$. Optimal $\tau^a = .07629$ at $T = 3$.
\end{itemize}

\newpage
\section{How does Optimal T vary with political strength?}
Trying to understand what is really going on with constraint in terms of $T$
\begin{itemize}
	\item Want to get good intuition for why $T$ can go up as $\sigma \downarrow$.
		\begin{itemize}
			\item Not obvious that it always does; I think it's possible that the direction of $T$ in response to $\sigma$ is indeterminant
		\end{itemize}
	\item I think I need to show effect of $\sigma$ on $\ov{e}$ first (I have informally that $\sigma \uparrow \Rightarrow \ov{e} \downarrow$)
		\begin{itemize}
			\item Then, impact of $\sigma$ on $\tau^a$. Next look to see if net profits at $\tau^a$ increase more than those at $\tau^{tw}$, then lobby's future incentives are muted
		\end{itemize}
\end{itemize}

\vskip.2in
\begin{result}
  $\frac{\mathrm{d} \ov{e}}{\mathrm{d} \sigma} > 0$
\end{result}

Proof: Corollary 4 shows that $\frac{\mathrm{d} \ov{e}}{\mathrm{d} \ga} < 0$. All that is left is to show that $\frac{\mathrm{d} \ga}{\mathrm{d} \sigma} < 0$.
\begin{itemize}
	\item The derivative of $\ga = 1 + \frac{1}{\sigma}e^\sigma$ w.r.t. $\sigma$ is
			\[
			  \frac{1}{\sigma} \ln \sigma e^\sigma + e^\sigma \left( -\frac{1}{\sigma^2}\right)
			\]
			Both terms are negative given $\sigma \in \left(0,1\right)$ and $e\geq 0$. QED.
\end{itemize}

\vskip.2in
Now for the result on $\tau^a$. Differentiating the lobby's condition with respect to $\sigma$, we have
\[
  \frac{\partial \Pi}{\partial \tau^a}\frac{\mathrm{d} \tau^a}{\mathrm{d} \ga}\frac{\mathrm{d} \ga}{\mathrm{d} \sigma} + \frac{\partial \Pi}{\partial \ov{e}}\frac{\mathrm{d} \ov{e}}{\mathrm{d} \ga}\frac{\mathrm{d} \ga}{\mathrm{d} \sigma} + \frac{\partial \Pi}{\partial \ga}\frac{\mathrm{d} \ga}{\mathrm{d} \sigma} = 0
\]

\begin{equation}
 	\frac{\mathrm{d} \tau^a}{\mathrm{d} \ga} \frac{\mathrm{d} \ga}{\mathrm{d} \sigma} = \left[-\frac{\frac{\partial \Pi}{\partial \ov{e}}\frac{\mathrm{d} \ov{e}}{\mathrm{d} \ga}+\frac{\partial \Pi}{\partial \ga}}{\frac{\partial \Pi}{\partial \tau^a}} \right] \frac{\mathrm{d} \ga}{\mathrm{d} \sigma}
\end{equation}
We know from Corollary 5 that $\frac{\mathrm{d} \tau^a}{\mathrm{d} \ga}$ is positive, and we've just shown that $\frac{\mathrm{d} \ga}{\mathrm{d} \sigma}$ is negative. Thus $\frac{\mathrm{d} \tau^a}{\mathrm{d} \sigma} <0$.

\vskip1in
Write constraint:
\[
  \ov{e}(\tau^a) - \pi(\tau^b(\ov{e}(\tau^a))) + \pi(\tau^a) - e_a - \frac{\de_\text{L} - \de_\text{L}^{T+1}}{1-\de_\text{L}} \left[\pi(\tau^{tw}) - e_{tw} -\pi(\tau^a) + e_a \right] = 0
\]

For now, assume this has an interior solution so calculus works. \\

First, what does $T$ do to $\ov{e}(\tau^a)$?

By the Implicit Function Theorem:
\begin{equation}
 	\frac{\mathrm{d} \ov{e}}{\mathrm{d} T} = -\frac{\frac{\partial \Omega}{\partial T}}{\frac{\partial \Omega}{\partial \ov{e}}} = 
	- \textstyle \frac{- \frac{\de_\text{ML}^{T+1}\ln\de_\text{ML}}{1-\de_\text{ML}}\left[  W_\text{ML}(\ga(\ov{e}),\bta) - W_\text{ML}(\ga(\ov{e}),\btw) \right]} {\frac{\de_\text{ML} - \de_\text{ML}^{T+1}}{1-\de_\text{ML}}\frac{\partial \ga}{\partial \ov{e}}\left[ \pi(\tau^a) - \pi(\tau^{tw}) \right] - \frac{\partial \ga}{\partial \ov{e}}\left[ \pi(\tau^b(\ov{e})) - \pi(\tau^{a}) \right]} > 0
	\label{eq:coret}
\end{equation}
So if $T \uparrow$ then $\ov{e} \uparrow$.

\vskip.2in
Now, want to know about effect of $T$ on $\tau^a$.
\[
  \frac{\partial \Pi}{\partial \tau^a}\frac{\mathrm{d} \tau^a}{\mathrm{d} T} + \frac{\partial \Pi}{\partial \ov{e}}\frac{\mathrm{d} \ov{e}}{\mathrm{d} T} + \frac{\partial \Pi}{\partial T} = 0
\]
Because $\frac{\partial \Pi}{\partial T} = \frac{\ln \de_\text{L} \de_\text{L}^{T+1}}{1-\de_\text{L}}\left[\pi(\tau^{tw}) - e_{tw} - \pi(\tau^a) + e_a\right]$, we are looking for
\begin{equation}
 	\frac{\mathrm{d} \tau^a}{\mathrm{d} T} = -\frac{\frac{\partial \Pi}{\partial \ov{e}}\frac{\mathrm{d} \ov{e}}{\mathrm{d} T} + \frac{\partial \Pi}{\partial T}}{\frac{\partial \Pi}{\partial \tau^a}} = 
	\frac{-\left(1 - \frac{\mathrm{d} \pi}{\mathrm{d} \ov{e}}\right) \cdot \frac{\mathrm{d} \ov{e}}{\mathrm{d} T} -\frac{\ln \de_\text{L} \de_\text{L}^{T+1}}{1-\de_\text{L}}\left[\pi(\tau^{tw}) - e_{tw} - \pi(\tau^a) + e_a\right]}{\left(1 + \frac{\de_\text{L} - \de_\text{L}^{T+1}}{1-\de_\text{L}}\right)\left[\frac{\partial \pi(\tau^a)}{\partial \tau^a} - \frac{\partial e_a}{\partial \tau^a}\right]}
\end{equation}
The proof of Corollary 3 shows that $\left(1 - \frac{\mathrm{d} \pi}{\mathrm{d} \ov{e}}\right)$ is positive, and the above result shows that $\frac{\mathrm{d} \ov{e}}{\mathrm{d} T}$ is positive. The second term is positive since net profits are maximized at $e_{tw}$ and $\de_L < 1$ so that its log is negative. With the leading negative signs, the numerator has both a negative and a positive part. This is not changed by the denominator, as the arguments given in the proof of Corollary 1 show that the denominator is positive. 
\begin{itemize}
	\item Remember simplified version where actors are infinitely patient
	\item Can play around with trick of looking for minimum $\tau^a$ by setting this equal to zero
\end{itemize}

\vskip1in
As $\sigma \downarrow, \ \ov{e} \downarrow$ for a given $\tau^a$.
\begin{itemize}
	\item Remember that change in $\ov{e}$ is really a change in the ($\tau^a,\ov{e}(\tau^a)$) schedule that derives from legislature's constraint
	\item So $\tau^a$ has to be raised to satisfy lobby's constraint
		\begin{itemize}
			\item Net profits are greatest at $\tau^{tw}$, so relative gap between net profits at $\tau^a$ and $\tau^{tw}$ (future) closes faster than that between break profits and trade agreement profits (present)
		\end{itemize}
	\item How much $\tau^a$ adjusts depends on magnitude of $\frac{\de_\text{L} + \de_\text{L}^{T+1}}{1-\de_\text{L}}$
	\item $\pi(\tau^b(\ov{e}(\tau^a))) - \ov{e}(\tau^a)$ is negative, gets less negative when $\ov{e}$ is reduced.
	\item $\pi(\tau^a) - e_a$ is positive, becomes larger as $\tau^a$ rises
\end{itemize}
\[
  0 \geq -\left[\ov{e}(\tau^a) - \pi(\tau^b(\ov{e}(\tau^a))) + \pi(\tau^a) - e_a \right] + \frac{\de_\text{L} - \de_\text{L}^{T+1}}{1-\de_\text{L}} \left[\pi(\tau^{tw}) - e_{tw} -\pi(\tau^a) + e_a \right]
\]
Where present part of constraint in on left and future part is on right. Remember present part must be negative.


\vskip1in
Let $\ga(e) = 1 + \frac{1}{1-\ta}e^{1-\ta}$. Or $\ga(e) = 1 + \frac{1}{\sigma}e^{\sigma}$.
\begin{itemize}
	\item $\frac{\partial \ga}{\partial e} = e^{\sigma - 1} > 0 \ \forall \sigma$
	\item $\frac{\partial^2 \ga}{\partial \sigma \partial e} = \ln e \cdot e^{\sigma - 1} < 0$ for $e<1$
\end{itemize}
i.e. as $\sigma \downarrow$, $\frac{\partial \ga}{\partial e} \uparrow$.


\vskip.5in
Can I show that the optimal $T$ can go either way when $\sigma$ changes? That is, a counterexample to my quasi-result?
\begin{itemize}
	\item My result says that as lobby gets stronger ($\frac{\partial \ga}{\partial e} \uparrow$, so $\sigma \downarrow$), $T$ should have to decrease.
	\item If optimal $T$ increases when $\sigma$ decreases (i.e. if $T$ is decreasing in $\sigma$), this is a counterexample.
		\begin{itemize}
			\item I think it's possible that both cases can happen depending on other parameters, like $\de$.
		\end{itemize}
\end{itemize}

Look at legislature's constraint:
\begin{multline*}
  \frac{\de_\text{ML} - \de_\text{ML}^{T+1}}{1-\de_\text{ML}} \left[W_\text{ML}(\ga(e_b),\bta) - W_\text{ML}(\ga(e_b),\btw) \right] \geq \\
	W_\text{ML}(\ga(e_b),\tau^b(e_b),\tau^{*a}) - W_\text{ML}(\ga(e_b),\bta)
\end{multline*}
If $T$ is too small, future gap can be smaller than current-period gap. So if $T$ gets too short, can't enforce on legislature.

\vskip1in
Note that changing $\sigma$ changes $\tau^{tw}$
\begin{itemize}
	\item $e_{tw}$ is solution to $\frac{\partial \pi}{\partial \tau}\frac{\partial \tau}{\partial \ga}\frac{\partial \ga}{\partial e} = 1$; or $\frac{\partial \pi}{\partial \tau}\frac{\partial \tau}{\partial \ga} = \frac{1}{\frac{\partial \ga}{\partial e}}$
	\item When $\frac{\partial \ga}{\partial e} \uparrow$, RHS $\downarrow$, so LHS must go down.
\end{itemize}

\newpage
\section{Optimal T result with perfect patience}
November 13, 2015
\begin{itemize}
	\item I know that $\frac{\partial \tau^a}{\partial T}$ has both a negative and positive part.
		\begin{itemize}
			\item Shown for the general case where $\de$ is anything.
		\end{itemize}
	\item Now I want to know if/when $\frac{\partial^2 \tau^a}{\partial T^2}$ is positive so that I could set $\frac{\partial \tau^a}{\partial T} = 0$ and optimal T (the one that minimizes $\tau^a$)
		\begin{itemize}
			\item Would need to be careful of corner solutions
			\item Will do this for case where $\de \rightarrow 1$
				\begin{itemize}
					\item This means $\frac{\de - \de^{T+1}}{1-\de} \rightarrow T$ and $- \frac{\de^{T+1}\ln\de}{1-\de} \rightarrow 1$
				\end{itemize}
		\end{itemize}
\end{itemize}

\vskip.2in
Start with simplifying result for $\frac{\mathrm{d} \tau^a}{\mathrm{d} T}$. \\ Again:
\begin{equation}
  \frac{\partial \Pi}{\partial \tau^a}\frac{\mathrm{d} \tau^a}{\mathrm{d} T} + \frac{\partial \Pi}{\partial \ov{e}}\frac{\mathrm{d} \ov{e}}{\mathrm{d} T} + \frac{\partial \Pi}{\partial T} = 0
	\label{eq:totalderiva}
\end{equation}
 
\begin{equation}
 	\frac{\mathrm{d} \tau^a}{\mathrm{d} T} = -\frac{\frac{\partial \Pi}{\partial \ov{e}}\frac{\mathrm{d} \ov{e}}{\mathrm{d} T} + \frac{\partial \Pi}{\partial T}}{\frac{\partial \Pi}{\partial \tau^a}} = 
	\frac{-\left(1 - \frac{\mathrm{d} \pi}{\mathrm{d} \ov{e}}\right) \cdot \frac{\mathrm{d} \ov{e}}{\mathrm{d} T} + \left[\pi(\tau^{tw}) - e_{tw} - \pi(\tau^a) + e_a\right]}{\left(1 + T\right)\left[\frac{\partial \pi(\tau^a)}{\partial \tau^a} - \frac{\partial e_a}{\partial \tau^a}\right]}
\end{equation}
\begin{itemize}
	\item $\left(1 - \frac{\mathrm{d} \pi}{\mathrm{d} \ov{e}}\right)$ is positive
	\item $\frac{\mathrm{d} \ov{e}}{\mathrm{d} T}$ is positive:
	  \begin{equation}
			\frac{\mathrm{d} \ov{e}}{\mathrm{d} T} = -\frac{\frac{\partial \Omega}{\partial T}}{\frac{\partial \Omega}{\partial \ov{e}}} = 
	\textstyle \frac{  W_\text{ML}(\ga(\ov{e}),\bta) - W_\text{ML}(\ga(\ov{e}),\btw)} {T\frac{\partial \ga}{\partial \ov{e}}\left[ \pi(\tau^{tw})-\pi(\tau^a) \right] + \frac{\partial \ga}{\partial \ov{e}}\left[ \pi(\tau^b(\ov{e})) - \pi(\tau^{a}) \right]} > 0
			\label{eq:coretsimple}
		\end{equation}
	\item denominator is positive (proof of Corollary 1)
\end{itemize}

\vskip.2in
Now, on to $\frac{\mathrm{d}^2 \tau^a}{\mathrm{d} T^2}$
\[
  \frac{\partial}{\partial T}\left[\frac{\partial \Pi}{\partial \tau^a}\frac{\mathrm{d} \tau^a}{\mathrm{d} T} + \frac{\partial \Pi}{\partial \ov{e}}\frac{\mathrm{d} \ov{e}}{\mathrm{d} T} + \frac{\partial \Pi}{\partial T} \right] = 0
\]
\begin{equation}
  \frac{\partial \Pi}{\partial \tau^a}\frac{\mathrm{d}^2 \tau^a}{\mathrm{d} T^2} + \frac{\partial^2 \Pi}{\partial T\partial \tau^a}\frac{\mathrm{d} \tau^a}{\mathrm{d} T} + \frac{\partial \Pi}{\partial \ov{e}} \frac{\mathrm{d}^2 \ov{e}}{\mathrm{d} T^2}+ \frac{\partial^2 \Pi}{\partial T \partial \ov{e}} \frac{\mathrm{d} \ov{e}}{\mathrm{d} T} + \frac{\partial^2 \Pi}{\partial T^2} = 0
	\label{eq:total2ndderiva}
\end{equation}

Going to need:
\[
  \frac{\partial^2 \Pi}{\partial T\partial \tau^a} = \frac{\partial \pi(\tau^a)}{\partial \tau^a} - \frac{\partial e_a}{\partial \tau^a} > 0
\]
\[
  \frac{\partial^2 \Pi}{\partial T \partial \ov{e}} = 0
\]
\[
	\frac{\partial^2 \Pi}{\partial T^2} = 0
\]
To get $\frac{\mathrm{d}^2 \ov{e}}{\mathrm{d} T^2}$, have to do a little more work.
\[
  \frac{\partial}{\partial T} \left[\frac{\partial \Omega}{\partial \ov{e}}\frac{\mathrm{d} \ov{e}}{\mathrm{d} T} +\frac{\partial \Omega}{\partial T} \right]= 0
\]
\[
  \frac{\partial \Omega}{\partial \ov{e}}\frac{\mathrm{d}^2 \ov{e}}{\mathrm{d} T^2} + \frac{\partial^2 \Omega}{\partial T \partial \ov{e}}\frac{\mathrm{d} \ov{e}}{\mathrm{d} T} +\frac{\partial^2 \Omega}{\partial T^2} = 0
\]
\[
  \frac{\partial^2 \Omega}{\partial T^2} = \frac{\partial}{\partial T} \left(\frac{\partial \Omega}{\partial T} \right) = 0
\]
\[
  \frac{\partial^2 \Omega}{\partial T \partial \ov{e}} = \frac{\partial}{\partial T} \left( \frac{\partial \Omega}{\partial \ov{e}}\right) = \frac{\partial \ga}{\partial \ov{e}}\left[ \pi(\tau^a)-\pi(\tau^{tw}) \right] < 0
\]

\vskip.2in
So,
\[
  \frac{\mathrm{d}^2 \ov{e}}{\mathrm{d} T^2} = \frac{- \frac{\partial^2 \Omega}{\partial T \partial \ov{e}}\frac{\mathrm{d} \ov{e}}{\mathrm{d} T} -\frac{\partial^2 \Omega}{\partial T^2} }{\frac{\partial \Omega}{\partial \ov{e}}} = \frac{- \frac{\partial^2 \Omega}{\partial T \partial \ov{e}}\frac{\mathrm{d} \ov{e}}{\mathrm{d} T} }{\frac{\partial \Omega}{\partial \ov{e}}} < 0
\]

\vskip.2in
Now we have everything we need for $\frac{\mathrm{d}^2 \tau^a}{\mathrm{d} T^2}$. Rearranging Equation~\ref{eq:total2ndderiva}, we have
\[
  \frac{\mathrm{d}^2 \tau^a}{\mathrm{d} T^2} = - \frac{ \frac{\partial^2 \Pi}{\partial T\partial \tau^a}\frac{\mathrm{d} \tau^a}{\mathrm{d} T} + \frac{\partial \Pi}{\partial \ov{e}} \frac{\mathrm{d}^2 \ov{e}}{\mathrm{d} T^2}+ \frac{\partial^2 \Pi}{\partial T \partial \ov{e}} \frac{\mathrm{d} \ov{e}}{\mathrm{d} T} + \frac{\partial^2 \Pi}{\partial T^2}}{\frac{\partial \Pi}{\partial \tau^a}}
\]
Elements of the last two terms in the denominator have been shown to be zero when $\de \rightarrow 1$, so we have
\[
  \frac{\mathrm{d}^2 \tau^a}{\mathrm{d} T^2} = - \frac{ \frac{\partial^2 \Pi}{\partial T\partial \tau^a}\frac{\mathrm{d} \tau^a}{\mathrm{d} T} + \frac{\partial \Pi}{\partial \ov{e}} \frac{\mathrm{d}^2 \ov{e}}{\mathrm{d} T^2}}{\frac{\partial \Pi}{\partial \tau^a}}
\]
The denominator is positive, so as goes the numerator, so goes the whole thing. So let's write out the numerator:
\begin{multline}
  \left( \frac{\partial \pi(\tau^a)}{\partial \tau^a} - \frac{\partial e_a}{\partial \tau^a}\right)\left[\frac{\left(1 - \frac{\mathrm{d} \pi}{\mathrm{d} \ov{e}}\right) \cdot \frac{\mathrm{d} \ov{e}}{\mathrm{d} T} - \left[\pi(\tau^{tw}) - e_{tw} - \pi(\tau^a) + e_a\right]}{\left(1 + T\right)\left[\frac{\partial \pi(\tau^a)}{\partial \tau^a} - \frac{\partial e_a}{\partial \tau^a}\right]} \right] - \\ \left(1 - \frac{\mathrm{d} \pi}{\mathrm{d} \ov{e}}\right) \frac{\partial \ga}{\partial \ov{e}}\left[ \pi(\tau^{tw})-\pi(\tau^a) \right] \frac{W_\text{ML}(\ga(\ov{e}),\bta) - W_\text{ML}(\ga(\ov{e}),\btw)}{\left[T\frac{\partial \ga}{\partial \ov{e}}\left[ \pi(\tau^{tw})-\pi(\tau^a) \right] + \frac{\partial \ga}{\partial \ov{e}}\left[ \pi(\tau^b(\ov{e})) - \pi(\tau^{a}) \right]\right]^2}
\end{multline}

\begin{multline}
  \frac{\left(1 - \frac{\mathrm{d} \pi}{\mathrm{d} \ov{e}}\right) \cdot \frac{\mathrm{d} \ov{e}}{\mathrm{d} T} - \left[\pi(\tau^{tw}) - e_{tw} - \pi(\tau^a) + e_a\right]}{\left(1 + T\right)} - \\ \frac{\left(1 - \frac{\mathrm{d} \pi}{\mathrm{d} \ov{e}}\right) \frac{\partial \ga}{\partial \ov{e}}\left[ \pi(\tau^{tw})-\pi(\tau^a) \right] \left[W_\text{ML}(\ga(\ov{e}),\bta) - W_\text{ML}(\ga(\ov{e}),\btw)\right]}{\left[T\frac{\partial \ga}{\partial \ov{e}}\left[ \pi(\tau^{tw})-\pi(\tau^a) \right] + \frac{\partial \ga}{\partial \ov{e}}\left[ \pi(\tau^b(\ov{e})) - \pi(\tau^{a}) \right]\right]^2}
\end{multline}
		
\vskip1in
\subsection{Solutions for optimal T}
On the solution(s) to $\frac{\mathrm{d} \tau^a}{\mathrm{d} T}$:
\begin{itemize}
	\item $\frac{\mathrm{d}^2 \tau^a}{\mathrm{d} T^2}$ has one term that is positive, one term that reverses the sign of $\frac{\mathrm{d} \tau^a}{\mathrm{d} T}$
	\item this means if $\frac{\mathrm{d} \tau^a}{\mathrm{d} T} =0$, it can't be a global max
		\begin{itemize}
			\item The slope would have to start positive, end negative.
			\item The second term in $\frac{\mathrm{d}^2 \tau^a}{\mathrm{d} T^2}$ would start negative, end positive, so whole term would be positive as $T$ gets large
			\item This means SOC CAN'T be negative everywhere. A contradiction
		\end{itemize}
	\item It's perfectly consistent to be a global min
		\begin{itemize}
			\item Let slope start negative, end positive.
			\item The second term in $\frac{\mathrm{d}^2 \tau^a}{\mathrm{d} T^2}$ would start positive, end negative, so whole term would start positive. As $T$ gets large, either stay positive (global min), or turn negative
			\item If start positive and turn negative, would be an inflection point and slope doesn't actually turn positive
		\end{itemize}
	\item If $\frac{\mathrm{d}^2 \tau^a}{\mathrm{d} T^2}$ positive everywhere and $\frac{\mathrm{d} \tau^a}{\mathrm{d} T}\left(0\right) < 0$ then interior min
	\item We know the negative part of $\frac{\mathrm{d} \tau^a}{\mathrm{d} T}$ gets smaller as $T \uparrow$
		\begin{itemize}
			\item If negative to start out, eventually becomes positive
			\item If positive to start out, gets more positive. 
				\begin{itemize}
					\item In this case, want $T$ as small as possible.
				\end{itemize}
		\end{itemize}
\end{itemize}

\vskip.2in
In line with last point, must fully explore corner solutions
\begin{itemize}
	\item What is TRUE? (not just what would be convenient for me)
\end{itemize}

\section[Cross Partial w.r.t. Sigma]{Cross Partial w.r.t. \texorpdfstring{$\sigma$}}
\begin{equation}
  \frac{\partial \Pi}{\partial \tau^a}\frac{\mathrm{d}^2 \tau^a}{\mathrm{d} \sigma \mathrm{d} T} + \frac{\partial^2 \Pi}{\partial \sigma \partial \tau^a}\frac{\mathrm{d} \tau^a}{\mathrm{d} T} + \frac{\partial \Pi}{\partial \ov{e}} \frac{\mathrm{d}^2 \ov{e}}{\mathrm{d} \sigma \mathrm{d} T}+ \frac{\partial^2 \Pi}{\partial \sigma \partial \ov{e}} \frac{\mathrm{d} \ov{e}}{\mathrm{d} T} + \frac{\partial^2 \Pi}{\partial \sigma \partial T} = 0
\end{equation}
\begin{itemize}
	\item Solving for second term
	\item Already know first (+), fourth (probably convex), fifth (+), eighth (+)
	\item Leaves four new terms to solve for. Third, sixth, seventh, ninth.
\end{itemize}

\vskip.2in
Ninth. Need:
\[
	\frac{\partial}{\partial \sigma} \left[\frac{\partial \Pi}{\partial T}\right] = \frac{\partial}{\partial \sigma} \left[ - \left( \pi(\tau^{tw}) - e_{tw} -\pi(\tau^a) + e_a \right) \right]
\]
\begin{itemize}
	\item $\sigma$ changes $\tau^{tw}$
	\item leave $\tau^a$ alone as it's an equilibrium object
	\item also changes how much lobby has to pay for $\tau^{tw}$ and $\tau^a$


\vskip.2in
\item Conjecture:
\begin{itemize}
	\item When $\sigma \uparrow$, the lobby weakens $\left(\frac{\partial \gamma}{\partial e} \ \downarrow \right)$. This  means $\pi(\tau^{tw}) - e_{tw} \ \downarrow$
	\item So probably overall gap goes down because lobby can't get as much in the trade war
		\begin{itemize}
			\item i.e. when lobby gets weaker, its future gain is less because it can't exert as much influence in the future (this makes sense)
		\end{itemize}
	\item there's a second effect on the trade war tariff payment that mirrors the tariff cap payment $\left( \frac{\partial e_{tw}}{\partial \ga}\frac{\partial \ga}{\partial \sigma} \right)$. If this is smaller in magnitude than the matching effect on $e_a$ (I hope this is true because $e_{tw}$ is closer to the max point and therefore the top of the envelope), then there's no problem to show the whole term is positive.
\end{itemize}
\end{itemize}

\vskip.5in
Third. Need:
\[
	\frac{\partial}{\partial \sigma} \left[\frac{\partial \Pi}{\partial \tau^a}\right] = \frac{\partial}{\partial \sigma} \left[ \left(1 + T\right)\left(\frac{\partial \pi(\tau^a)}{\partial \tau^a} - \frac{\partial e_a}{\partial \tau^a}\right) \right]
\]
\begin{itemize}
	\item For any given $\tau$, $\sigma$ doesn't affect $\pi(\tau)$.
	\item But it \textit{does} affect how much has to paid to get $\tau$.
	\item When $\sigma \ \uparrow$ , have to pay more and more. This is the first derivative.
	\item As $\tau^a \ \uparrow$, $e_a \ \uparrow$, so $-e_a \ \downarrow$. As $\sigma \ \uparrow$, $e_a \ \uparrow$ further, so $-e_a \ \downarrow$ further. So this term must be negative overall.
\end{itemize}

\vskip.5in
Seventh. Note that
\[
  \frac{\partial \Pi}{\partial \ov{e}} = 1 - \frac{\partial \pi}{\partial \ov{e}} = 1 - \frac{\partial \pi}{\partial \tau}\frac{\partial \tau^b}{\partial \ga}\frac{\partial \ga}{\partial \ov{e}}
\]	
Need
\[
	\frac{\partial}{\partial \sigma} \left[\frac{\partial \Pi}{\partial \ov{e}}\right] = \frac{\partial}{\partial \sigma} \left[ 1 - \frac{\partial \pi}{\partial \tau}\frac{\partial \tau^b}{\partial \ga}\frac{\partial \ga}{\partial \ov{e}} \right] = -  \frac{\partial}{\partial \sigma} \left[ \frac{\partial \pi}{\partial \tau}\frac{\partial \tau^b}{\partial \ga}\frac{\partial \ga}{\partial \ov{e}} \right]
\]
\[
  = \frac{\partial \pi}{\partial \tau} \left[\frac{\partial \tau^b}{\partial \ga}\frac{\partial^2 \ga}{\partial \sigma \partial \ov{e}} + \frac{\partial^2 \tau^b}{\partial \sigma \partial \ga}\frac{\partial \ga}{\partial \ov{e}}\right] + \frac{\partial^2 \pi}{\partial \sigma \partial \tau} \frac{\partial \tau^b}{\partial \ga}\frac{\partial \ga}{\partial \ov{e}}
\]
\begin{itemize}
	\item Since $\sigma$ is not involved in how $\tau$ maps to profits, $\frac{\partial^2 \pi}{\partial \sigma \partial \tau} =0$ and the second additive term is zero
	\item We know that $\frac{\partial \pi}{\partial \tau} > 0$, so the sign of the seventh term is determined by the sign of what's inside the square brackets x $(-1)$
	\item $\frac{\partial^2 \tau^b}{\partial \sigma \partial \ga} = 0$: once $\ga$ is determined, it maps directly into $\tau$ given the form of the welfare function. $\sigma$ only controls how $e$ maps into $\ga$.
	\item So we're left with $\frac{\partial \tau^b}{\partial \ga}\frac{\partial^2 \ga}{\partial \sigma \partial \ov{e}}$. We know the first term is positive, so just need to investigate $\frac{\partial^2 \ga}{\partial \sigma \partial \ov{e}}$.
		\begin{itemize}
			\item $\frac{\partial^2 \ga}{\partial \sigma \partial e} = \ln e \cdot e^{\sigma - 1} < 0$ for $e<1$
			\item This means that the positive slope becomes larger as $\sigma \uparrow$. That is, the influence increases
			\item $\frac{\partial}{\partial \sigma} \left[ \frac{\partial\pi}{\partial e} \right] < 0$, so $\frac{\partial}{\partial \sigma} \left[ 1 - \frac{\partial\pi}{\partial e} \right] > 0$
		\end{itemize}
\end{itemize}

\vskip.5in
Sixth
\[
  \frac{\mathrm{d}^2 \ov{e}}{\mathrm{d} \sigma \mathrm{d} T}
\]
\[
	\frac{\mathrm{d} \ov{e}}{\mathrm{d} T} = -\frac{\frac{\partial \Omega}{\partial T}}{\frac{\partial \Omega}{\partial \ov{e}}} = \frac{  W_\text{ML}(\ga(\ov{e}),\bta) - W_\text{ML}(\ga(\ov{e}),\btw)}{T\frac{\partial \ga}{\partial \ov{e}}\left[ \pi(\tau^{tw})-\pi(\tau^a) \right] + \frac{\partial \ga}{\partial \ov{e}}\left[ \pi(\tau^b(\ov{e})) - \pi(\tau^{a}) \right]} > 0
\]
\[
  \frac{\partial}{\partial \sigma} \left[\frac{\partial \Omega}{\partial \ov{e}}\frac{\mathrm{d} \ov{e}}{\mathrm{d} T} +\frac{\partial \Omega}{\partial T} \right]= 0
\]
\[
  \frac{\partial \Omega}{\partial \ov{e}}\frac{\mathrm{d}^2 \ov{e}}{\mathrm{d} \sigma \mathrm{d} T} + \frac{\partial^2 \Omega}{\partial \sigma \partial \ov{e}}\frac{\mathrm{d} \ov{e}}{\mathrm{d} T} +\frac{\partial^2 \Omega}{\partial \sigma \partial T} = 0
\]
We're solving for $\frac{\mathrm{d}^2 \ov{e}}{\mathrm{d} \sigma \mathrm{d} T}$. Know that $\frac{\partial \Omega}{\partial T}$ is positive.
\begin{itemize}
	\item Know that $\frac{\mathrm{d} \ov{e}}{\mathrm{d} T}$ is positive.
	\item Need two terms:
		\begin{itemize}
			\item $\frac{\partial^2 \Omega}{\partial \sigma \partial T} = \frac{\partial}{\partial \sigma} \left(\frac{\partial \Omega}{\partial T} \right) = \frac{\partial}{\partial \sigma}\left(T\frac{\partial \ga}{\partial \ov{e}}\left[ \pi(\tau^a)-\pi(\tau^{tw}) \right] + \frac{\partial \ga}{\partial \ov{e}}\left[ \pi(\tau^{a}) - \pi(\tau^b(\ov{e})) \right] \right)$
			\[
			  = T\frac{\partial^2 \ga}{\partial \sigma \partial \ov{e}}\left[ \pi(\tau^a)-\pi(\tau^{tw}) \right] - T \frac{\partial \ga}{\partial \ov{e}} \frac{\partial \pi(\tau^{tw})}{\partial \sigma}+ \frac{\partial^2 \ga}{\partial \sigma \partial \ov{e}}\left[ \pi(\tau^{a}) - \pi(\tau^b(\ov{e})) \right] > 0
			\]
			Every term is negative except the $T$ and the $\frac{\partial \ga}{\partial \ov{e}}$, so the whole expression is positive.
			\item $\frac{\partial^2 \Omega}{\partial \sigma \partial \ov{e}} = \frac{\partial}{\partial \sigma}\left(\frac{\partial \Omega}{\partial \ov{e}}\right) = \frac{\partial}{\partial \sigma}\left( W_\text{ML}(\ga(\ov{e}),\bta) - W_\text{ML}(\ga(\ov{e}),\btw)\right)$
				\begin{itemize}
					\item Get $\frac{\partial \ga}{\partial \sigma}\left[\pi_x(\tau^a) - \pi_x(\tau^{tw})\right]$ term, which is negative times negative so positive
					\item And get $-\frac{\mathrm{d}}{\mathrm{d} \sigma} W_\text{ML}(\ga(\ov{e}),\btw)$ term, which is overall negative. But can be split into social welfare part and $(\ga-1)\left(-\frac{\partial\pi_x(\tau^{tw})}{\partial \tau^{tw}}\frac{\partial \tau^{tw}}{\partial \ga}\frac{\partial \ga}{\partial \sigma} \right)$ part. First one is clearly positive (remember, $\sigma \uparrow$ makes $\gamma \downarrow$), second is negative (these include leading negative sign).
					\item Concavity of $\pi(\cdot)$ implies that $\frac{\partial\pi_x(\tau^{tw})}{\partial \tau^{tw}} \geq \pi_x(\tau^{tw}) - \pi_x(\tau^a)$. Can use this to show that the whole term is positive as long as
						\[
						  \pi_x(\tau^a) - \pi_x(\tau^{tw}) \geq \frac{\partial\pi_x(\tau^{tw})}{\partial \tau^{tw}} \frac{\partial \tau^{tw}}{\partial \ga} (\ga - 1)
						\]
				\end{itemize}
		\end{itemize}
	\item As long as $\frac{\partial^2 \Omega}{\partial \sigma \partial \ov{e}} \geq 0$, $\frac{\mathrm{d}^2 \ov{e}}{\mathrm{d} \sigma \mathrm{d} T} > 0$.
\end{itemize}

\vskip.2in
What is intuition? 
\begin{itemize}
	\item Intuition for $\frac{\partial \ov{e}}{\partial T}$ result: as $T \ \uparrow$, the legisltaure doesn't want to break the agreement. So $\ov{e}$ has to be raised to make it indifferent again because that makes it value profits more: make current gain larger through increased weight on profits AND through larger $\tau^b$; make future loss smaller by increasing $\ga$.
		\begin{itemize}
			\item Now, if $\sigma \uparrow$, lobby gets weaker, so legislature cares less about profits, all else equal.
				\begin{itemize}
					\item Will have to increase $e$ by even more to get a big enough increase in $\ga$ and $\tau^b$.
					\item BUT have to be careful, because $\tau^{tw}$ also falls, which reduces the future loss and somewhat reduces the required $e$ to create indifference. Without some conditions, it's possible that this effect could be stronger (at least I haven't ruled it out).
				\end{itemize}
			\item Alternatively, from the point of view of $\frac{\partial}{\partial T}\frac{\partial \ov{e}}{\partial \sigma}$: as lobby gets weaker, cutoff $e$ goes up. And the increase in $\ov{e}$ gets larger as $T \ \uparrow$
		\end{itemize}
\end{itemize}

\vskip1in
Now putting it all together
\begin{itemize}
	\item If $\ga \uparrow$, $T^*$ should get smaller.
		\begin{itemize}
			\item Requires slope of $\frac{\partial \tau^a}{\partial T}$ to increase when $\ga$ increases
			\item i.e. $\frac{\mathrm{d}^2 \tau^a}{\mathrm{d} \ga \mathrm{d} T} > 0$
		\end{itemize}
	\item If $\sigma \uparrow$, $T^*$ should get larger.
		\begin{itemize}
			\item Because $\sigma \uparrow \Rightarrow \ga \downarrow$
			\item Requires slope of $\frac{\partial \tau^a}{\partial T}$ to decrease when $\sigma$ increases
			\item i.e. $\frac{\mathrm{d}^2 \tau^a}{\mathrm{d} \sigma \mathrm{d} T} < 0$
		\end{itemize}

\end{itemize}

\[
  \frac{\mathrm{d}^2 \tau^a}{\mathrm{d} \sigma \mathrm{d} T} = \frac{- \frac{\partial^2 \Pi}{\partial \sigma \partial \tau^a}\frac{\mathrm{d} \tau^a}{\mathrm{d} T} - \frac{\partial \Pi}{\partial \ov{e}} \frac{\mathrm{d}^2 \ov{e}}{\mathrm{d} \sigma \mathrm{d} T} - \frac{\partial^2 \Pi}{\partial \sigma \partial \ov{e}} \frac{\mathrm{d} \ov{e}}{\mathrm{d} T} - \frac{\partial^2 \Pi}{\partial \sigma \partial T}}{\frac{\partial \Pi}{\partial \tau^a}}
\]
I have every individual term positive except $\frac{\partial^2 \Pi}{\partial \sigma \partial \tau^a}$ and $\frac{\mathrm{d} \tau^a}{\mathrm{d} T}$.
\begin{itemize}
	\item $\frac{\partial^2 \Pi}{\partial \sigma \partial \tau^a}$ is negative
	\item $\frac{\mathrm{d} \tau^a}{\mathrm{d} T}$ is probably convex. If so, taken with two negative signs, this makes the negative slope more negative and positive slope more positive. At any rate, everywhere the slope is negative becomes more negative.
\end{itemize}


\newpage
At some point, may want to verify via Young's theorem (take derivative of lobby's condition w.r.t. $\sigma$ first, then w.r.t. $T$):
\[
  \frac{\partial \Pi}{\partial \tau^a} \frac{\partial}{\partial T} \left[\frac{\partial \tau^a}{\partial \ga }\frac{\partial \ga}{\partial \sigma }\right] + \frac{\partial^2 \Pi}{\partial T \partial \tau^a} \left[\frac{\partial \tau^a}{\partial \ga }\frac{\partial \ga}{\partial \sigma }\right] + \frac{\partial \Pi}{\partial \ov{e}} \frac{\partial}{\partial T} \left[\frac{\partial \ov{e}}{\partial \ga }\frac{\partial \ga}{\partial \sigma }\right]+ \frac{\partial^2 \Pi}{\partial T \partial \ov{e}} \left[\frac{\partial \ov{e}}{\partial \ga }\frac{\partial \ga}{\partial \sigma}\right] + \frac{\partial \Pi}{\partial \ga}\frac{\partial^2 \ga}{\partial T\partial \sigma} + \frac{\partial^2 \Pi}{\partial T \partial \ga}\frac{\partial \ga}{\partial \sigma} = 0
\]


\end{document}