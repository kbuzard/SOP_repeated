\documentclass[12pt]{article}

\addtolength{\textwidth}{1.4in}
\addtolength{\oddsidemargin}{-.7in} %left margin
\addtolength{\evensidemargin}{-.7in}
\setlength{\textheight}{8.5in}
\setlength{\topmargin}{0.0in}
\setlength{\headsep}{0.0in}
\setlength{\headheight}{0.0in}
\setlength{\footskip}{.5in}
\renewcommand{\baselinestretch}{1.0}
\setlength{\parindent}{0pt}
\linespread{1.1}

\usepackage{amssymb, amsmath, amsthm, bm}
\usepackage{graphicx,csquotes,verbatim}
\usepackage[backend=biber,block=space,style=authoryear]{biblatex}
\setlength{\bibitemsep}{\baselineskip}
\usepackage[american]{babel}
%dell laptop
\addbibresource{C:/Users/Kristy/Dropbox/Research/xBibs/tradeagreements.bib}
%\addbibresource{C:/Users/Kristy/Documents/Dropbox/Research/xBibs/tradeagreements.bib}
\renewcommand{\newunitpunct}{,}
\renewbibmacro{in:}{}


\DeclareMathOperator*{\argmax}{arg\,max}
\usepackage{xcolor}
\hbadness=10000

\newcommand{\ve}{\varepsilon}
\newcommand{\ov}{\overline}
\newcommand{\un}{\underline}
\newcommand{\ta}{\theta}
\newcommand{\al}{\alpha}
\newcommand{\Ta}{\Theta}
\newcommand{\expect}{\mathbb{E}}
\newcommand{\Bt}{B(\bm{\tau^a})}
\newcommand{\bta}{\bm{\tau^a}}
\newcommand{\btn}{\bm{\tau^n}}
\newcommand{\ga}{\gamma}
\newcommand{\Ga}{\Gamma}
\newcommand{\de}{\delta}

\begin{document}
\begin{center}
Miscellaneous notes on JIE R$\&$R of SOP$\_$Repeated
\end{center}

\vskip.3in
Legislative constraint as a function of $e$
\begin{itemize}
	\item I thought it would be positive at $e=0$ and turn negative as $e$ increases
	\item What does it mean that for some values it's negative at 0, becomes positive, and then goes negative again?
		\begin{itemize}
			\item For sure I have to be careful in numerical examples
		\end{itemize}
\end{itemize}

\newpage
Numerical examples
\begin{table*}
	\centering
		\begin{tabular}{crrrrr}
			      & E=.35 & E=.4 						& E=.41						& E=.42						& E=.45\\
			$\tau^tw$	&		&									&									& .074						& .0654\\		
			$e^tw$  	&		&									&									&     						& .00123\\		
			T = 2 &       & .07500 					&									&									& .057407\\
			T = 3 &       & .074716 				& .070243					& .066284					& \textbf{.0570802}\\
			T = 4 &       & \textbf{.074708}& \textbf{.070233}& \textbf{.066275}& .0570806\\
			T = 5 &       & .074795 				& .07033 					& .06638					& .057185\\
			T = 6 & .1080 & .07492 & &&\\
			T = 7 & .1081 & 			 					& 								&									& .057\\
			T = 8 & .10814 & 			 & & &
		\end{tabular}
	\caption{$\de_L = \de_{ML} =.95$}
	\label{tab:d95}
\end{table*}
\end{document}