\documentclass[12pt]{article}

\addtolength{\textwidth}{1.4in}
\addtolength{\oddsidemargin}{-.7in} %left margin
\addtolength{\evensidemargin}{-.7in}
\setlength{\textheight}{8.5in}
\setlength{\topmargin}{0.0in}
\setlength{\headsep}{0.0in}
\setlength{\headheight}{0.0in}
\setlength{\footskip}{.5in}
\renewcommand{\baselinestretch}{1.0}
\setlength{\parindent}{0pt}
\linespread{1.1}

\usepackage{amssymb, amsmath, amsthm, bm}
\usepackage{graphicx,csquotes,verbatim}
\usepackage[backend=biber,block=space,style=authoryear]{biblatex}
\setlength{\bibitemsep}{\baselineskip}
\usepackage[american]{babel}
%dell laptop
\addbibresource{C:/Users/Kristy/Dropbox/Research/xBibs/tradeagreements.bib}
%\addbibresource{C:/Users/Kristy/Documents/Dropbox/Research/xBibs/tradeagreements.bib}
\renewcommand{\newunitpunct}{,}
\renewbibmacro{in:}{}


\DeclareMathOperator*{\argmax}{arg\,max}
\usepackage{xcolor}
\hbadness=10000

\newcommand{\ve}{\varepsilon}
\newcommand{\ov}{\overline}
\newcommand{\un}{\underline}
\newcommand{\ta}{\theta}
\newcommand{\al}{\alpha}
\newcommand{\Ta}{\Theta}
\newcommand{\expect}{\mathbb{E}}
\newcommand{\Bt}{B(\bm{\tau^a})}
\newcommand{\bta}{\bm{\tau^a}}
\newcommand{\btn}{\bm{\tau^n}}
\newcommand{\btw}{\bm{\tau^{tw}}}
\newcommand{\ga}{\gamma}
\newcommand{\Ga}{\Gamma}
\newcommand{\de}{\delta}

\newtheorem{result}{Result}

\begin{document}
\begin{center}
Miscellaneous notes on JIE R$\&$R of SOP$\_$Repeated
\end{center}

\vskip.3in
Legislative constraint as a function of $e$
\begin{itemize}
	\item I thought it would be positive at $e=0$ and turn negative as $e$ increases
	\item What does it mean that for some values it's negative at 0, becomes positive, and then goes negative again?
		\begin{itemize}
			\item For sure I have to be careful in numerical examples
		\end{itemize}
\end{itemize}

\newpage
\begin{center}
Numerical examples 
\end{center}
$\de_L = \de_{ML} =.95$ \\
		\begin{tabular}{crrrrr}
			      & E=.35 & E=.4 						& E=.41						& E=.42						& E=.45\\
			$\tau^{tw}$	&		&									&									& .074						& .0654\\		
			$e^{tw}$  	&		&									&									&     						& .00123\\		
			T = 2 &       & .07500 					&									&									& .057407\\
			T = 3 &       & .074716 				& .070243					& .066284					& \textbf{.0570802}\\
			T = 4 &       & \textbf{.074708}& \textbf{.070233}& \textbf{.066275}& .0570806\\
			T = 5 &       & .074795 				& .07033 					& .06638					& .057185\\
			T = 6 & .1080 & .07492 & &&\\
			T = 7 & .1081 & 			 					& 								&									& .057\\
			T = 8 & .10814 & 			 & & &
		\end{tabular}
	
\vskip.5in
I have another sheet of notes that conflicts with the first column. It just says ``$\de=.95$'':
		\begin{tabular}{cr}
			      & E=.35 \\
			$\tau^{tw}$	&	.1213	\\		
			$e^{tw}$  	&	.006003	\\		
			T = 3 & .1023044      \\
			T = 4 & \textbf{.1022411} \\
			T = 5 & .1022427      \\
			T = 6 & .10227 \\
			T = 7 & .102305 \\
		\end{tabular}
	
\vskip.5in
This one just says ``$\de=.99$'':\\
		\begin{tabular}{crr}
			      & E=.4 & .39\\
			$\tau^{tw}$	&	&	\\		
			$e^{tw}$  	&	&	\\		
			T = 2 & .068510 & .06391      \\
			T = 3 & \textbf{.068261} & \textbf{.06374}      \\
			T = 4 & .068289 & .06378      \\
			T = 5 & .06841 & .06391      
		\end{tabular}

\vskip.5in
This has the note, ``This at least works in the direction I thought it would'' with ``$\de_L=.94, \ \de_{ML}=.95$'':\\
		\begin{tabular}{cr}
			      & E=.4 \\
			$\tau^{tw}$		&	\\		
			$e^{tw}$  		&	\\		
			T = 4 & .07464 \\
			T = 5 & \textbf{.07421} \\
			T = 6 & .07481      \\
			T = 7 & .07492       
		\end{tabular} \\
(``Really want to know if reducing $\de_L$ --- making future term less important --- will give me the $\sigma$ result I've been after; really, no result at all; depends on other parameters.)

\vskip.5in
Some summaries
\begin{itemize}
	\item $E=.4$, $\de_L=.99, \ \de_{ML}=.95$, $e_{tw} = .00232$, $\tau^{tw} = .08185$. Optimal $\tau^a = .07494$ at $T = 3$.
	\item $E=.5$, assume I kept $\de_L=.99, \ \de_{ML}=.95$. Optimal $\tau^a = .04864$ at $T = 3$.
	\item $E=.4$, $\de_L=\de_{ML}=.99$, $e_{tw} = .00232$, $\tau^{tw} = .08185$. Optimal $\tau^a = .07470$ at $T = 3$.
	\item $E=.4$, $\de_L=.99, \ \de_{ML}=.5$, $e_{tw} = .00232$, $\tau^{tw} = .08185$. Optimal $\tau^a = .07802$ at $T = 2$.
	\item $E=.4$, $\de_L=.99, \ \de_{ML}=.75$, $e_{tw} = .00232$, $\tau^{tw} = .08185$. Optimal $\tau^a = .07629$ at $T = 3$.
\end{itemize}

\newpage
Trying to understand what is really going on with constraint in terms of $T$
\begin{itemize}
	\item Want to get good intuition for why $T$ can go up as $\sigma \downarrow$.
		\begin{itemize}
			\item Not obvious that it always does; I think it's possible that the direction of $T$ in response to $\sigma$ is indeterminant
		\end{itemize}
	\item I think I need to show effect of $\sigma$ on $\ov{e}$ first (I have informally that $\sigma \uparrow \Rightarrow \ov{e} \downarrow$)
		\begin{itemize}
			\item Then, impact of $\sigma$ on $\tau^a$. Next look to see if net profits at $\tau^a$ increase more than those at $\tau^{tw}$, then lobby's future incentives are muted
		\end{itemize}
\end{itemize}

\begin{result}
  $\frac{\mathrm{d} \ov{e}}{\mathrm{d} \sigma} > 0$
\end{result}

Proof: Corollary 4 shows that $\frac{\mathrm{d} \ov{e}}{\mathrm{d} \ga} < 0$. All that is left is to show that $\frac{\mathrm{d} \ga}{\mathrm{d} \sigma} < 0$.
\begin{itemize}
	\item The derivative of $\ga = 1 + \frac{1}{\sigma}e^\sigma$ w.r.t. $\sigma$ is
			\[
			  \frac{1}{\sigma} \ln \sigma e^\sigma + e^\sigma \left( -\frac{1}{\sigma^2}\right)
			\]
			Both terms are negative given $\sigma \in \left(0,1\right)$ and $e\geq 0$. QED.
\end{itemize}


\vskip1in
Write constraint:
\[
  \ov{e}(\tau^a) - \pi(\tau^b(\ov{e}(\tau^a))) + \pi(\tau^a) - e_a - \frac{\de_\text{L} + \de_\text{L}^{T+1}}{1-\de_\text{L}} \left[\pi(\tau^{tw}) - e_{tw} -\pi(\tau^a) + e_a \right] = 0
\]

For now, assume this has an interior solution so calculus works. \\

First, what does $T$ do to $\ov{e}(\tau^a)$?

By the Implicit Function Theorem:
\begin{equation}
 	\frac{\mathrm{d} \ov{e}}{\mathrm{d} T} = -\frac{\frac{\partial \Omega}{\partial T}}{\frac{\partial \Omega}{\partial \ov{e}}} = 
	\textstyle \frac{- \frac{\de_\text{ML}^{T+1}\ln\de_\text{ML}}{1-\de_\text{ML}}\left[  W_\text{ML}(\ga(\ov{e}),\bta) - W_\text{ML}(\ga(\ov{e}),\btw) \right]} {\frac{\de_\text{ML} - \de_\text{ML}^{T+1}}{1-\de_\text{ML}}\frac{\partial \ga}{\partial \ov{e}}\left[ \pi(\tau^a) - \pi(\tau^{tw}) \right] - \frac{\partial \ga}{\partial \ov{e}}\left[ \pi(\tau^b(\ov{e})) - \pi(\tau^{a}) \right]} > 0
	\label{eq:coret}
\end{equation}
So if $T \uparrow$ then $\ov{e} \uparrow$.

\vskip1in
(Again, assuming) as $\sigma \uparrow, \ \ov{e} \downarrow$ for a given $\tau^a$.
\begin{itemize}
	\item So $\tau^a$ has to be raised to satisfy lobby's constraint
		\begin{itemize}
			\item Net profits are greatest at $\tau^{tw}$, so relative gap between net profits at $\tau^a$ and $\tau^{tw}$ (future) closes faster than that between break profits and trade agreement profits (present)
		\end{itemize}
	\item How much $\tau^a$ adjusts depends on magnitude of $\frac{\de_\text{L} + \de_\text{L}^{T+1}}{1-\de_\text{L}}$
	\item $\pi(\tau^b(\ov{e}(\tau^a))) - \ov{e}(\tau^a)$ is negative, gets less negative when $\ov{e}$ is reduced.
	\item $\pi(\tau^a) - e_a$ is positive, becomes larger as $\tau^a$ rises
\end{itemize}
\[
  0 \geq -\left[\ov{e}(\tau^a) - \pi(\tau^b(\ov{e}(\tau^a))) + \pi(\tau^a) - e_a \right] + \frac{\de_\text{L} + \de_\text{L}^{T+1}}{1-\de_\text{L}} \left[\pi(\tau^{tw}) - e_{tw} -\pi(\tau^a) + e_a \right]
\]
Where present part of constraint in on left and future part is on right. Remember present part must be negative.


\vskip1in
Let $\ga(e) = 1 + \frac{1}{1-\ta}e^{1-\ta}$. Or $\ga(e) = 1 + \frac{1}{\sigma}e^{\sigma}$.
\begin{itemize}
	\item $\frac{\partial \ga}{\partial e} = e^{\sigma - 1} > 0 \ \forall \sigma$
	\item $\frac{\partial^2 \ga}{\partial \sigma \partial e} = \ln e \cdot e^{\sigma - 1} < 0$ for $e<1$
\end{itemize}
i.e. as $\sigma \downarrow$, $\frac{\partial \ga}{\partial e} \uparrow$.


\vskip.5in
Can I show that the optimal $T$ can go either way when $\sigma$ changes? That is, a counterexample to my quasi-result?
\begin{itemize}
	\item My result says that as lobby gets stronger ($\frac{\partial \ga}{\partial e} \uparrow$, so $\sigma \downarrow$), $T$ should have to decrease.
	\item If optimal $T$ increases when $\sigma$ decreases (i.e. if $T$ is decreasing in $\sigma$), this is a counterexample.
		\begin{itemize}
			\item I think it's possible that both cases can happen depending on other parameters, like $\de$.
		\end{itemize}
\end{itemize}

Look at legislature's constraint:
\begin{multline*}
  \frac{\de_\text{ML} - \de_\text{ML}^{T+1}}{1-\de_\text{ML}} \left[W_\text{ML}(\ga(e_b),\bta) - W_\text{ML}(\ga(e_b),\btw) \right] \geq \\
	W_\text{ML}(\ga(e_b),\tau^b(e_b),\tau^{*a}) - W_\text{ML}(\ga(e_b),\bta)
\end{multline*}
If $T$ is too small, future gap can be smaller than current-period gap. So if $T$ gets too short, can't enforce on legislature.

\vskip1in
Note that changing $\sigma$ changes $\tau^{tw}$
\begin{itemize}
	\item $e_{tw}$ is solution to $\frac{\partial \pi}{\partial \tau}\frac{\partial \tau}{\partial \ga}\frac{\partial \ga}{\partial e} = 1$; or $\frac{\partial \pi}{\partial \tau}\frac{\partial \tau}{\partial \ga} = \frac{1}{\frac{\partial \ga}{\partial e}}$
	\item When $\frac{\partial \ga}{\partial e} \uparrow$, RHS $\downarrow$, so LHS must go down.
\end{itemize}
\end{document}