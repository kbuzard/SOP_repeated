%C:\Users\Kristy\Dropbox\Research\SOP\SOP_repeated\JIE_RR
\documentclass[12pt]{article}

\addtolength{\textwidth}{1.4in}
\addtolength{\oddsidemargin}{-.7in} %left margin
\addtolength{\evensidemargin}{-.7in}
\setlength{\textheight}{8.5in}
\setlength{\topmargin}{0.0in}
\setlength{\headsep}{0.0in}
\setlength{\headheight}{0.0in}
\setlength{\footskip}{.5in}
\renewcommand{\baselinestretch}{1.0}
\setlength{\parindent}{0pt}
\linespread{1.1}

\usepackage[pdftex,
bookmarks=true,
bookmarksnumbered=false,
pdfview=fitH,
bookmarksopen=true]{hyperref}

\usepackage{amssymb, amsmath, amsthm, bm}
\usepackage{graphicx,csquotes,verbatim}
\usepackage[backend=biber,block=space,style=authoryear]{biblatex}
\setlength{\bibitemsep}{\baselineskip}
\usepackage[american]{babel}
%dell laptop
\addbibresource{C:/Users/Kristy/Dropbox/Research/xBibs/tradeagreements.bib}
%\addbibresource{C:/Users/Kristy/Documents/Dropbox/Research/xBibs/tradeagreements.bib}
\renewcommand{\newunitpunct}{,}
\renewbibmacro{in:}{}


\DeclareMathOperator*{\argmax}{arg\,max}
\usepackage{xcolor}
\hbadness=10000

\newcommand{\ve}{\varepsilon}
\newcommand{\ov}{\overline}
\newcommand{\un}{\underline}
\newcommand{\ta}{\theta}
\newcommand{\al}{\alpha}
\newcommand{\Ta}{\Theta}
\newcommand{\expect}{\mathbb{E}}
\newcommand{\Bt}{B(\bm{\tau^a})}
\newcommand{\bta}{\bm{\tau^a}}
\newcommand{\btn}{\bm{\tau^n}}
\newcommand{\btw}{\bm{\tau^{tw}}}
\newcommand{\ga}{\gamma}
\newcommand{\Ga}{\Gamma}
\newcommand{\de}{\delta}

\newtheorem{result}{Result}

\begin{document}
\begin{center}
Notes on 2nd round of revisions for JIE of SOP$\_$Repeated
\end{center}

\vskip.3in
\section{General notes}
There are three sets of comments from the editor and referees on which I'd like to ask your feedback.
\begin{enumerate}
	\item The first is a matter of how the model is to be interpreted: whether there are really two branches of government, or one branch of government with time-inconsistent preferences. Alternatively, as the editor suggests, one branch of government with time-consistent preferences that encounters different levels of lobbying at different stages.
	\item The second is related to the first: the analysis should proceed differently depending on the interpretation.
	\item The third is whether I can cast my model as a special case of Dixit, Grossman and Helpman (1997), as we discussed at an earlier meeting. I believe the first two points are related in important ways to to this point.
\end{enumerate}

\vskip.5in
On 3:
In my model, I argue that the break tariff $(\tau^b(\ov{e}))$ should be chosen according to the lobbying effort applied in the break stage $(\ov{e})$ because this determines the identity of the median legislator. The median legislator in the current period gets to pick the tariff in the current period.
\begin{itemize}
	\item DGH has a different interpretation. A unitary government is presented with a schedule of $(\tau,e)$ pairs and gets to pick which pair it would like.
		\begin{itemize}
			\item The editor has suggested that I can choose a very simple take-it-or-leave-it schedule with one just pair or $e=0$, but in order for my calculus-based results to hold up, I think I need a differentiable schedule.
			\item At any rate, no one in the government gets to 'pick' $\tau$ on its own; a particular $\tau$ is demanded in return for a particular $e$. In other words, the gov't makes a decision about which $e$ to take and which $\tau$ to apply by virtue of which $e$ it takes.
			\item In my model, who the decision-maker IS is determined by the decision of the lobby as to how much $e$ to put forth
		\end{itemize}
	\item Is it possible to have a DGH-style indifference condition in the spirit of my model, i.e. $e$ determines the median legislator?
		\begin{itemize}
			\item In a normal repeated game, a player chooses current period action to maximize his whole stream of payoffs.
			\item But that's not quite what happens in my model because the current player will not be making decisions in the future. So the current player makes his decision with a view to influencing who gets to make the decision in the future.
			\item Current player finds $\tau$ that maximizes current payoffs, then checks to see whether it will also maximize future payoffs. If it doesn't switches to trade agreement tariff.
				\begin{itemize}
					\item That is, the maximization program for current player is complex: has to find optimal $\tau$ if he's going to break, then compare total payoff stream under $\tau^b(\ov{e}))$ to total payoff under $\tau^a$
				\end{itemize}
		\end{itemize}
\end{itemize}

\vskip1in
Some possibilities for indifference that don't really work
\begin{enumerate}
	\item The legislature will get one of two choices
		\begin{itemize}
			\item If breaking is too expensive, $(\tau^{opt},0)$ vs. $(\tau^a,e_a)$
				\begin{itemize}
					\item Can I calculate this just like trade war pair?
				\end{itemize}
			\item If breaking is worthwhile, $(\tau^{opt},0)$ vs. $(\tau^b,\ov{e})$
				\begin{itemize}
					\item See equation in section 4 below
				\end{itemize}
		\end{itemize}
\end{enumerate}
Whatever condition I come up with for determining $\ov{e}$, need to see if I can say whether $\ov{e} \geq \leq e_{tw}$, how $\ov{e}$ compares to one from base model for a given $\tau^b$ at the level the lobby actually choses
\begin{itemize}
	\item 
\end{itemize}

\newpage
\section{Interpretation: SOP vs. time inconsistency}
\textbf{Referee 1 Point 1}: \\
I believe the interpretation of the model as representing two branches of the government is arbitrary. It is more realistic to interpret the model as representing the preferences of a government with time-inconsistent preferences, such that ex ante the government's objective is to maximize social welfare, but at the time of implementing the agreement it might be influenced by lobby group activities. \\

I understand that the author has tried to modify the interpretation of the model to address this concern, but in my view the modification was insufficient. \\

\vskip.2in
\textbf{Editor Point 5}:\\
Regarding Referee 1's idea about time-inconsistent preferences, I guess I am ok with your separation-of-powers interpretation (subject to the cavets I expressed in my first-round letter), but I also think it would be useful to discuss a possible alternative interpretation along the lines of Referee 1's idea. I am thinking of the Maggi and Rodriguez-Clare setting, where there is a unitary government that cares about welfare and contributions, but at the stage of signing the agreement ("ex ante" stage) the lobby has no influence on the government, because the lobby cares only about the short run (due to the fact that specific capital is mobile in the long run). The way I think about this interpretation is slightly different from Referee 1, in that there is no time inconsistency of preferences: the government has the same preferences across time, but ex-ante the lobby is not active. I am not sure whether a Maggi and Rodriguez-Clare specification of this kind would yield similar results as yours. This might be an interesting question to discuss, and if the answer is yes, it would be useful to point out that the model admits also this alternative interpretation.
\begin{itemize}
	\item Note this also connects to analysis section
	\item To generalize Giovanni's 'government has the same preferences across time, but ex-ante the lobby is not active' idea, ex-post, the lobby can be active at different levels and this does not change the 'preferences,' which are really the process by which the median legislator is chosen.
\end{itemize}


\newpage
\section{Continuation payoffs and changes in the preferences}
\textbf{Editor Comment 1}: \\
I agree with Referee 2 that there are still problems with the analysis. Correcting these problems is a necessary (though not sufficient) condition for me to move forward with this paper. You will need to convince us beyond the reasonable doubt that the analysis is correct. \\

\textbf{Editor Point 8}:\\
The way you write the key program in (9)-(11) is confusing because you have an ``e" floating around, and it is not clear where it should be evaluated. Unless I am missing something, in some places this should be $\ov{e}(\tau^a)$ and in other places it should be $e^a$. It would also be helpful to write $e^a$ as a function of $tau^a$. Since the choice variable in the program is $\tau^a$, you should make clear what is a function of $\tau^a$ and what is not. \\

\textbf{Referee 1 Point 3}: \\
It is assumed that the current legislator evaluates future welfare (i.e., the continuation payoffs) based on the expected preferences of the future government, which will be induced by lobbying efforts in the future. Alternatively, it could be assumed that the current legislator evaluates future profits based on its current preferences. \\

The latter assumption might be more consistent with the premise of the model, which is essentially a decision-making model with time-inconsistent preferences. Moreover, I think the results related to self-enforceability of the agreement will continue to hold if the author adopts the latter assumption. \\

Some discussion of this point could be illuminating. \\

\textbf{Referee 2 Point 3}: \\
 I am confused about one basic aspect of the incentive constraint for the legislature. Consider the RHS of (7) on page 16. If the lobby chooses e and the legislature decides to select its best response given e, $\tau^R(e)$, and thereby breaks the agreement, then a T-period punishment is launched in the next period. The notation in (7) (and likewise in (10)) suggests that the lobby continues choosing the same e, thus generating the same gamma(e), during the punishment phase. I don't see why this would be the case. Wouldn't the lobby instead choose the effort level $e_{tw}$ that is determined in the first-order condition given by (6)? And indeed, if we look at the lobby incentive constraint, as given by the RHS of (8) (and likewise in (11))), we see that that constraint does assume that $e_{tw}$ is used by the lobby during the punishment phase. I can't tell exactly what is going on here. There could be an oversight, or I could be misunderstanding the notation. At a minimum, some clarification is needed. \\

This consideration also leads to a further concern/question. If my point above is correct and the lobby should be modeled in (7) as choosing $e_{tw}$ in the punishment phase, and if $e_{tw} > e$, then would the legislature ever deviate (even for e in the non-triggering range as currently defined) in order to trigger a trade war and thereby enjoy the higher $e_{tw}$ and thus the higher gamma that the trade war elicits? Recall that $W_{ML}$ is increasing in e as an independent argument. Is this potential incentive captured? \\

\textbf{Referee 2 Point v}: \\
Page 20: Related to comment 3 above, I don't follow why in (12) that e can't change from e-bar as we move into the trade war.

\vskip.5in
\un{My response}:\\
Under the alternative suggested by Referee 2, Equation (7) would be modified from 
\begin{multline}
  W_\text{ML}(\ga(e),\bta) + \frac{\de_\text{ML} - \de_\text{ML}^{T+1}}{1-\de_\text{ML}} W_\text{ML}(\ga(e),\bta) \geq \\
	W_\text{ML}(\ga(e),\tau^R(e),\tau^{*a}) + \frac{\de_\text{ML} - \de_\text{ML}^{T+1}}{1-\de_\text{ML}} W_\text{ML}(\ga(e),\btw).
\end{multline}
to
\begin{multline}
  W_\text{ML}(\ga(e),\bta) + \frac{\de_\text{ML} - \de_\text{ML}^{T+1}}{1-\de_\text{ML}} W_\text{ML}(\ga(e^a),\bta) \geq \\
	W_\text{ML}(\ga(e),\tau^R(e),\tau^{*a}) + \frac{\de_\text{ML} - \de_\text{ML}^{T+1}}{1-\de_\text{ML}} W_\text{ML}(\ga(e^{tw}),\btw).
\end{multline}

That is, the current-period median legislator would evaluate the current-period incentive constraint using a mixture of his own weight on import-competing profits and those of the legislators who are median in the future in the trade agreement and trade-war scenarios.
\begin{itemize}
	\item What if you don't like the median legislator interpretation and want to think of it as a unitary decision-maker who faces different amounts of lobbying effort?
		\begin{itemize}
			\item Think about what would happen in a simpler repeated game...
		\end{itemize}
\end{itemize}

\newpage
\section{Dixit, Grossman and Helpman analogy}
\textbf{Editor Point 3}: \\
You state that your specification of legislature preferences can be seen as a special case of the Dixit-Grossman-Helpman (DGH) model. It would be helpful if you could substantiate this claim. More specifically, let us consider a simple version of the DGH specification in your setting. Suppose you modify your model only in two ways. First, suppose the legislature preferences are given by W+g(e), where g(e) is an increasing and concave function of contributions (while the lobby preferences remain the same). And second, suppose the lobby offers a contribution schedule e(t) before the legislature chooses the tariff. The question is: would this model deliver the same results as yours (at least qualitatively)? Note that in this setting you can focus on a simple all-or-nothing contribution schedule, of the kind "if you give me a 5$\%$ tariff I give you $\$$100, otherwise you get nothing." Intuitively the promised contribution will just compensate the legislator for the loss associated with the requested tariff, so the analysis might not be hard.  If this DGH version of your model yields similar qualitative results, pointing out this "isomorphism" would help you in several ways. First, it would provide "foundations" for your assumed legislature preferences, in terms of a model (DGH) that people are familiar with. Second, this would help address my question 2 above: in the DGH model we can think of e as money, and I think it would be reasonable to stick with your current definition of aggregate welfare (even though in principle one might question this definition of aggregate welfare when utility is not transferrable). And third, you could examine whether your results rely on the presence of diminishing marginal utility from contributions: what would happen if g is linear (as in the basic Protection for Sale model) rather than strictly concave? \\

\vskip.5in
DGH97 paper:
\begin{itemize}
	\item Has to be a truthful contribution schedule
		\begin{itemize}
			\item Proposition 3: $G(a^0,e^T(a^0,u^0)) = \max_a G(a,0)$
			\item $e^T(a^0,u^0)$ is essentially $\phi(a^0,u^0)$, which is defined implicitly in eqn3 (p. 760) as
				\[
				  U\left[ a, \phi(a^0,u^0)\right] = u^0
				\]
					\begin{itemize}
						\item Careful: truthful contribution schedule is not a best response function. But the lobby will still have to be best responding in equilibrium.
						\item Truthful contribution schedule is a device for solving equilibrium in their model. Doesn't mean I have to follow it (I think; I hope)
					\end{itemize}
			\item if $G = W + g(e)$ and $e=0$ and $g(0) = 0$, then $a^* = \tau^{\text{opt}}$
			\item Then rewrite as 
				\[
				  G(\tau^0,e^T(\tau^0,u^0)) = \max_a G(\tau^{\text{opt}},0)
				\]
					\begin{itemize}
						\item Right hand side provides a number
						\item Ignoring arguments of $e^T$ function, LHS traces out $\left(\tau,e\right)$ pairs that satisfy the equation given $g(e)$.
						\item For lobby to be best responding, it MUST pick the pair that maximizes $\pi(\tau) - e$. \textit{This} concern must be what sets $u^0$.
					\end{itemize}
		\end{itemize}
	\item Corollary to Prop 1 / Prop 3: Gov't gets utility equal to outside option. Is this true when there is just one lobby?
	\item Combining the two previous facts (if true in my case), then it must be that gov't getting outside option will set $u^0$ (eqm utility) and anchor contribution schedule (just have to be careful of zero contributions)
\end{itemize}

\vskip.2in
What editor proposes:
\begin{itemize}
	\item Lobby offers contribution schedule (can be very simple: just one $(e,\tau)$ pair, $e=0$ for everything else)
	\item Government maximizes $W + g(e)$
		\begin{itemize}
			\item Note that $CS_X + \ga(e) \cdot PS_X + CS_Y + PS_Y +TR = W + \left( \ga(e) - 1 \right) PS_X$
		\end{itemize}
	\item Sectioning from my paper:
		\begin{itemize}
			\item[3.1] Same (execs)
			\item[3.2] Trade war
				\begin{itemize}
					\item Government evaluates welfare (unilateral, essentially, because $\tau^*$ doesn't change) at the TIOLI offer of the lobby and at the value $\left(\tau^{opt}\right)$ that satisfies $\frac{\partial W}{\partial \tau} + \frac{\partial g(0}{\partial \tau} = 0$ [i think; this just means unilateral maximization with $e=0$ since nothing the leg does here will change the contribution]. Chooses which one maximizes welfare.
						\begin{itemize}
							\item How to think about lobby's contribution schedule?
							\item DGH Proposition 1: principal (lobby) has to provide at least agent's (gov'ts) outside option; as long as this constraint is satisfied, lobby can propose $\tau$ and payment that maximizes his own utility
							\item DGH Proposition 3 (p. 760-61): simplifies so we don't need to look for contribution functions, only eqm values
							\item Assume lobby has all the bargaining power as in DGH97. Then lobby calculates $\left(\tau,e\right)$ schedule from
								\[
									W(\tau) +g(e) = W(\tau^{opt}) +g(0)
								\]
								Assume $g(0) = 0$. Then
								\[
									g(e) = W(\tau^{opt}) - W(\tau)
								\]
								\[
									e = g^{-1}\left[W(\tau^{opt}) - W(\tau)\right]
								\]
						\end{itemize}
				\end{itemize}
			\item[3.3] Nothing seems to change
			\item[3.4] Nothing seems to change
			\item[4] I think the break phase is essentially the same, but the way it is calculated is different
				\begin{itemize}
					\item Lobby first determines $\ov{e}$, then decides whether it's worthwhile paying $\ov{e}$
					\item Need to re-write Equation 12, which defines $\ov{e}$
						\begin{itemize}	
							\item What \textit{is} $\ov{e}$? It's what the lobby has to pay to get the legislature to break the trade agreement
							\item I think I've been conceiving of it wrong. It's not necessarily the minimum $e$. Yes, $\ov{e}$ must satisfy (something like) Equation (7), but lobby may make higher profits choosing an $e$ higher than the minimum one in the base model. I don't know how to think about it in the DGH version yet.
							\item In base model, the $\tau$ for the current period is chosen in response to current period $e$; this decision is made differently (potentially) than the decision to break. Is there a similar separation between the decisions in the DGH version of the model?
							\item When lobby has all the bargaining power, there's a schedule $(\ov{e},\tau^b)$ that makes leg indifferent between breaking the trade agreement or abiding by it ($\tau^b$ no longer a direct fcn of $\ov{e}$)
								\begin{multline*}
									W(\bta) + g(0) + \frac{\de_\text{ML} - \de_\text{ML}^{T+1}}{1-\de_\text{ML}}\left[W(\bta) + g(0)\right] = \\
	W(\tau^b,\tau^{*a}) +g(e) + \frac{\de_\text{ML} - \de_\text{ML}^{T+1}}{1-\de_\text{ML}} \left[W(\btw) + g(e)\right] \ ?
								\end{multline*}
						\end{itemize}
				\end{itemize}
		\end{itemize}
\end{itemize}

\vskip.2in
Comparisons
\begin{itemize}
	\item Note that in GH94, claim is that IN EQUILIBRIUM, government behaves as if it maximizes a weighted sum of the groups' utilities ($1+a$ for those represented by lobbies; $a$ for those not represented; pg. 10 of PDF). But this is in equilibrium: there's a lot going on out of equilibrium that leads us there!
		\begin{itemize}
			\item I haven't proved this to myself, but that must also depend on lobbies having all the bargaining power
		\end{itemize}
	\item In DGH, $G(\tau^0,e^0) = \max_\tau (\tau,0)$
		\begin{itemize}
			\item This is \un{not} generally true in my model
			\item I think the slippage is in bargaining power. DGH paradigm assumes lobby essentially make TIOLI offer.
			\item $\ga(e)$ formulation in essence distributes bargaining power more generally
				\begin{itemize}
					\item Is it okay to characterize it this way?
					\item If so, does bargaining power vary with effort? Seems mixed up with diminishing returns to effort.
				\end{itemize}
			\item $W + \Phi(e)$ of Limao and Tovar gives decreasing returns; then explicitly models bargaining power through Nash bargain instead of menu auction (as far as I can tell--I can't find it clearly specified).
		\end{itemize}
	\item In DGH, essentially lobby chooses its favorite $(\tau,e)$ pair from among all the possible ones that make the Gov't indifferent.
		\begin{itemize}
			\item Is it possible that this is equivalent to $\frac{\partial \pi}{\partial e} = 1$?
			\item seems like the pair that satisfies that equation might not be available
			\item When I calculate the various $(\tau,e)$ pairs to 'compare' government welfare levels, this comes from the government welfare function: each $e$ induces an optimal $\tau$ and then I get an optimal value function
				\begin{itemize}
					\item Lobby doesn't care about these varying welfare levels. Chooses $(\tau,e)$ pair that maximizes $\pi - e$
				\end{itemize}
		\end{itemize}
\end{itemize}

\vskip.2in
Examples
\begin{itemize}
	\item Use numeric example from BS2005 in R (DGH.r)
	\item Isomorphism I've already calculated: if $G = W + e$ then $\ga(e) = 1 + \frac{e}{\pi_x}$
	\item Lobby's optimization:
		\[
		  W(\tau,e(\tau)) = W(0,0)
		\]
		\[
		  W(\tau) + e = W(0)
		\]
		\[
		  e = W(0) - W(\tau)
		\]
			
			\begin{itemize}
				\item This forms a contribution schedule
				\item Lobby chooses the pair $(\tau,e)$ that maximizes $\pi(\tau) - e$
			\end{itemize}
	\item Make sure these are unilateral changes in $\tau$ in the program
	\item Then compare to my old way: $\frac{\partial \pi}{\partial \tau} \frac{\partial \tau}{\partial \ga} \frac{\partial \ga}{\partial e} = 1$ ($e$ changes with $\tau$ directly depending on shape of $\ga$, then government maximizes w.r.t. $\tau$
		\begin{itemize}
			\item NEXT: go into R program and sort it out
		\end{itemize}
\end{itemize}

\vskip.5in
\subsection{Subtract effort in gov't welfare function}
\textbf{Editor Point 2}:\\
There is a question in my mind about the definition of welfare and the objective functions of the executive and the legislature. As I understand it, the lobbying effort involves a resource cost (this is not a cash transfer), and this cost in principle should be reflected in the expression for welfare. Viewed from a different perspective: you assume that the exceutive and the legislature care about import-competing profits, and thus about the lobby in a broad sense, but don't care about the resource cost of effort that the lobby incurs, and this seems hard to justify. I am sorry to be raising this issue at this stage and not in the previous round, but I became aware of it recently.

\newpage


\end{document}