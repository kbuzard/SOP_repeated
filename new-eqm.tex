\documentclass[12pt]{article}

\addtolength{\textwidth}{1.4in}
\addtolength{\oddsidemargin}{-.7in} %left margin
\addtolength{\evensidemargin}{-.7in}
\setlength{\textheight}{8.5in}
\setlength{\topmargin}{0.0in}
\setlength{\headsep}{0.0in}
\setlength{\headheight}{0.0in}
\setlength{\footskip}{.5in}
\renewcommand{\baselinestretch}{1.0}
\setlength{\parindent}{0pt}
\linespread{1.1}

\usepackage{amssymb, amsmath, amsthm, bm}
\usepackage{graphicx,csquotes,verbatim}
\usepackage[backend=biber,block=space,style=authoryear]{biblatex}
\setlength{\bibitemsep}{\baselineskip}
\usepackage[american]{babel}
%dell laptop
%\addbibresource{C:/Users/Kristy/Dropbox/Research/xBibs/tradeagreements.bib}
%\addbibresource{C:/Users/Kristy/Documents/Dropbox/Research/xBibs/tradeagreements.bib}
\renewcommand{\newunitpunct}{,}
\renewbibmacro{in:}{}


\DeclareMathOperator*{\argmax}{arg\,max}
\usepackage{xcolor}
\hbadness=10000

\newcommand{\ve}{\varepsilon}
\newcommand{\ov}{\overline}
\newcommand{\un}{\underline}
\newcommand{\ta}{\theta}
\newcommand{\al}{\alpha}
\newcommand{\Ta}{\Theta}
\newcommand{\expect}{\mathbb{E}}
\newcommand{\Bt}{B(\bm{\tau^a})}
\newcommand{\bta}{\bm{\tau^a}}
\newcommand{\btn}{\bm{\tau^n}}
\newcommand{\ga}{\gamma}
\newcommand{\Ga}{\Gamma}
\newcommand{\de}{\delta}

\begin{document}
\begin{center}
New Equilibrium Construction
\end{center}

From ``to\_do\_list.tex'': \\
\vskip.2in

Take out renegotiation
		\begin{itemize}
			\item Add more basic tradeoff
			\item (??) Draw inverted U for lobby
			\item Now my short punishments don't rest on renegotiation
				\begin{itemize}
					\item So now, for main analysis, must assume that we're constraining attention to a certain class of punishments: symmetric, and ``Punish for $T$ periods then go back to cooperation''
						\begin{itemize}
							\item Go back to start if deviate should work for governments, but I think I need something else for lobbies since they would like that
						\end{itemize}
					\item Can I show that mine are optimal in this class?
				\end{itemize}
		\end{itemize}

\vskip.5in
January 17, 2015
\begin{itemize}
	\item Must show players are best responding in every subgame, on and off the eqm path
	\item I'm going to try to use reversion to the static nash, but this is not necessarily subgame perfect (deviations can trigger changes in future periods)
		\begin{itemize}
			\item Basic intuition: lobby wants punishment to go longer, leg wants it to go shorter
			\item Ideally, want each to choose static BR in each period of punishment: in non-cooperative state, you can pick whatever you want, but the other guy is doing whatever he wants; $\tau^{tw}$ is independent of what he does
				\begin{itemize}
					\item BUT it's not independent of lobby's effort
				\end{itemize}
		\end{itemize}
\end{itemize}
	
\vskip.3in
Equilibrium: Executives set trade agreement at $t=0$. At $t \geq 1$, lobbies choose $e$, leg chooses applied $\tau$
\begin{itemize}
	\item $\forall \ t \geq 1$, leg applies $\tau^A$ if 
		\begin{enumerate}
			\item $\tau \leq \tau^A$ was applied last period
			\item There have been $T$ periods of punishment: I think $\tau \geq \tau^N$ and $e \leq e^N$
		\end{enumerate}
	\item Not sure how to specify lobby in these cooperation periods: $e = 0$ if $\tau \geq \tau^A$ (in any period? how are they involved in punishment? they're not really)
	\item if $\tau > \tau^A$ within the last $T$ periods, leg applies $\tau^N(e^N)$
\end{itemize}
	
\vskip.5in
January 19, 2015
\begin{itemize}
	\item Think of punishment scheme being designed either by execs or by supranational body like WTO
	\item Then want to know whether it's an eqm for leg and lobbies to follow the rules

\end{itemize}

\vskip.2in
Classes of subgames
\begin{enumerate}
	\item $\tau \leq \tau^A$ last period, $e=0$, and no violation within last $T$ periods (i.e. these conditions have held for at least $T$ previous periods)
	\item Conditions in (1) held for at least $T$ periods up to previous period, but there was a violation last period
		\begin{itemize}
			\item Implies we start punishing this period, and punish for $T-1$ more periods before switching back to (1)
		\end{itemize}
	\item A punishment was initiated within the last $T-1$ periods, and punishment has been followed since then
		\begin{itemize}
			\item Implies we punish for the rest of the duration of the specified punishment period (if there have been $i$ periods of punishment so far, we punish this period and $T-i-1$ more periods before switching back to (1)
		\end{itemize}
	\item Like 2, but someone did not follow the punishment in $t-1$
		\begin{itemize}
			\item If it was legislature and $\tau^D > \tau^N$, we don't care. (can describe punishment as $\tau \geq \tau^N$ I think)
				\begin{itemize}
					\item If it was leg and $\tau^D < \tau^N$, restart at (2)
				\end{itemize}
			\item If it was lobby and $e^D > e^N$, we don't care. (can describe punishment as $e \geq e^N$ I think)
				\begin{itemize}
					\item If it was leg and $e^D < e^N$, restart at (1)
				\end{itemize}
		\end{itemize}
	\item Like 3, but someone did not follow the punishment in $t-1$
		\begin{itemize}
			\item Same as 4, except constraints may differ by length of remaining punishment
				\begin{itemize}
					\item I've shown condition for lobby is constant through time except in last period, where they'll never pay
					\item Need to pay special attention to leg's condition in this last period
				\end{itemize}
		\end{itemize}
\end{enumerate}
	
	
\end{document}