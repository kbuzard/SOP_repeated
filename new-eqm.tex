\documentclass[12pt]{article}

\addtolength{\textwidth}{1.4in}
\addtolength{\oddsidemargin}{-.7in} %left margin
\addtolength{\evensidemargin}{-.7in}
\setlength{\textheight}{8.5in}
\setlength{\topmargin}{0.0in}
\setlength{\headsep}{0.0in}
\setlength{\headheight}{0.0in}
\setlength{\footskip}{.5in}
\renewcommand{\baselinestretch}{1.0}
\setlength{\parindent}{0pt}
\linespread{1.1}

\usepackage{amssymb, amsmath, amsthm, bm}
\usepackage{graphicx,csquotes,verbatim}
%\usepackage[backend=biber,block=space,style=authoryear]{biblatex}
%\setlength{\bibitemsep}{\baselineskip}
%\usepackage[american]{babel}
%dell laptop
%\addbibresource{C:/Users/Kristy/Dropbox/Research/xBibs/tradeagreements.bib}
%\addbibresource{C:/Users/Kristy/Documents/Dropbox/Research/xBibs/tradeagreements.bib}
%\renewcommand{\newunitpunct}{,}
%\renewbibmacro{in:}{}


\usepackage[pdftex,
bookmarks=true,
bookmarksnumbered=false,
pdfview=fitH,
bookmarksopen=true,hyperfootnotes=false]{hyperref}

\DeclareMathOperator*{\argmax}{arg\,max}
\usepackage{xcolor}
\hbadness=10000

\newcommand{\ve}{\varepsilon}
\newcommand{\ov}{\overline}
\newcommand{\un}{\underline}
\newcommand{\ta}{\theta}
\newcommand{\al}{\alpha}
\newcommand{\Ta}{\Theta}
\newcommand{\expect}{\mathbb{E}}
\newcommand{\Bt}{B(\bm{\tau^a})}
\newcommand{\bta}{\bm{\tau^a}}
\newcommand{\btn}{\bm{\tau^n}}
\newcommand{\ga}{\gamma}
\newcommand{\Ga}{\Gamma}
\newcommand{\de}{\delta}

\begin{document}
\begin{center}
New Equilibrium Construction
\end{center}

From ``to\_do\_list.tex'': \\
\vskip.2in

\section{From to-do list}
Take out renegotiation
		\begin{itemize}
			\item Add more basic tradeoff
			\item (??) Draw inverted U for lobby
			\item Now my short punishments don't rest on renegotiation
				\begin{itemize}
					\item So now, for main analysis, must assume that we're constraining attention to a certain class of punishments: symmetric, and ``Punish for $T$ periods then go back to cooperation''
						\begin{itemize}
							\item Go back to start if deviate should work for governments, but I think I need something else for lobbies since they would like that
						\end{itemize}
					\item Can I show that mine are optimal in this class?
				\end{itemize}
		\end{itemize}

\vskip.5in
\section{Equilibrium overview}	
January 17, 2015
\begin{itemize}
	\item Must show players are best responding in every subgame, on and off the eqm path
	\item I'm going to try to use reversion to the static nash, but this is not necessarily subgame perfect (deviations can trigger changes in future periods)
		\begin{itemize}
			\item Basic intuition: lobby wants punishment to go longer, leg wants it to go shorter
			\item Ideally, want each to choose static BR in each period of punishment: in non-cooperative state, you can pick whatever you want, but the other guy is doing whatever he wants; $\tau^{tw}$ is independent of what he does
				\begin{itemize}
					\item BUT it's not independent of lobby's effort
				\end{itemize}
		\end{itemize}
\end{itemize}


\vskip.3in
Equilibrium: Executives set trade agreement tariffs $\tau^A$ at $t=0$. At $t \geq 1$, lobbies choose $e$, leg chooses applied $\tau$
\begin{itemize}
	\item $\forall \ t \geq 1$, leg applies $\tau^A$ if 
		\begin{enumerate}
			\item $\tau \leq \tau^A$ was applied last period
			\item There have been $T$ periods of punishment: I think $\tau \geq \tau^N$ and $e \leq e^N$
		\end{enumerate}
	\item Not sure how to specify lobby in these cooperation periods: $e = 0$ if $\tau \geq \tau^A$ (in any period? how are they involved in punishment? they're not really)
	\item if $\tau > \tau^A$ within the last $T$ periods, leg applies $\tau^N(e^N)$
\end{itemize}
	
\vskip.5in
January 19, 2015
\begin{itemize}
	\item Think of punishment scheme being designed either by execs or by supranational body like WTO
	\item Then want to know whether it's an eqm for leg and lobbies to follow the rules

\end{itemize}

\vskip.2in
\section{Classes of subgames}
\begin{enumerate}
	\item $\tau \leq \tau^A$ and $e < \ov{e}$ last period; if there had ever been a violation, it ended at least $T$ periods previous.
		\begin{itemize}
			\item Play $\tau = \tau^A$ and $e=e_a$
		\end{itemize}
	\item Conditions in (1) held in period $t-2$, but there was a violation in period $t-1$
		\begin{itemize}
			\item Play static Nash this period and for $T-1$ more periods before switching back to (1); more precisely, $\tau^D \geq \tau^N$ and $e^D \geq e^N$.
		\end{itemize}
	\item Static Nash punishment was initiated $i<T$ periods ago, and punishment has been followed since then, i.e. $\tau^D \geq \tau^N$ and $e^D \geq e^N$ $\forall t \in$ (need to figure out indexing so I can say this precisely)
		\begin{itemize}
			\item Punish this period and $T-i-1$ more periods before switching back to (1)
		\end{itemize}
	\item In any punishment period (i.e. classes 2 and 3 above), legislature does not follow punishment: i.e. $\tau^D < \tau^N$
		\begin{itemize}
			\item Restart punishment at (2); lobby happy to do this
		\end{itemize}
	\item In any punishment period, lobby does not follow punishment: $e^D < e^N$
		\begin{itemize}
			\item Legislature chooses BR (make this precise: $\tau^D(e^D)) = ...$) to $e^D$, then restart at (1); if anything else, restart punishment from (2). BR + eqm, so okay
				\begin{itemize}
					\item Lobby \textit{must} pay in final period of punishment, or else IC for leg will not hold. That is why the equilibrium is being re-worked.
					\item But, if leg is going to BR to lobby's payment and then restart cooperation, lobby should want to continue with punishment. This seems like a realistic set-up (your protection ends if you don't hold up your end of the deal with the promised payments).
				\end{itemize}
		\end{itemize}
\end{enumerate}

\vskip.5in
\section{Conditions for equilibrium}
\begin{itemize}
	\item Checking that punishment (2) is incentive compatible given (4) and (5)
		\begin{itemize}
			\item Legislature:
				\[
				  W(\ga(e^N),\tau^N) + \frac{\de - \de^{T+1}}{1-\de}W(\ga(e^N),\tau^N) + \de^{T+1} W(\ga(0),\tau^a) \geq W(\ga(e^N),\cdot) + \frac{\de - \de^{T+1}}{1-\de}W(\ga(e^N),\tau^N)
				\]
				by definition, anything provides lower one-shot payoffs than $\tau^N$, and Nash payoffs are lower than trade agreement payoffs (need to prove this--or is it just by assumption?)
					\begin{itemize}
						\item Is this punishment IC for leg? Yes, condition is satisfied. Best responding in current period--would do the same in best deviation; future is trade agreement instead of restart of punishment for any other tariff.
						\item This conditions is always satisfied; the above equation is for the first period of the punishment. In later periods, there are more periods of the punishment that get repeated, so the constraint becomes looser
					\end{itemize}
			\item Lobby [(2) is incentive compatible given (4) and (5)]:
				\[
				  \pi(\tau^N) - e^N + \frac{\de - \de^{T+1}}{1-\de} \left[\pi(\tau^N) - e^N \right] + \de^{T+1} \pi(\tau^a) \geq \pi(\tau^D) -e^D + \frac{\de - \de^{T+2}}{1-\de} \pi(\tau^a) 
				\]
				best deviation, given that leg will one-shot best respond is also $e^N$; given $\pi(\tau^N) - e^N \geq \tau^a$, which is satisfied in equilibrium, this condition holds. \\
				Since the best deviation is to the Nash tariff, it reduces to
				\[
				  \frac{\de - \de^{T+1}}{1-\de} \left[\pi(\tau^N) - e^N \right] \geq \frac{\de - \de^{T+1}}{1-\de} \pi(\tau^a)
				\]
				\begin{equation}
				  \frac{\de - \de^{T+1}}{1-\de} \left[\pi(\tau^N) - e^N - \pi(\tau^a) \right] \geq 0
					\label{lobdev}
				\end{equation}
				This is always true in equilibrium. \\
				\vskip.2in
				Note that Expression~\ref{lobdev} will hold for all $T$ (or, alternatively, $i$).
					\begin{itemize}
						\item In last period of punishment, will hold with equality. Is tighest at the end of the punishment, looser at beginning. Holds for all $T$, i.e. always holds.
						\item Note that if a deviation were to occur (off eqm path), leg would apply a ``punishment'' tariff that is too low, which would normally cause the punishment to reset; but here, it's because the lobby deviated, so we re-start cooperation to punish the lobby further after giving the reduced tariff in the period in which the lobby deviates
					\end{itemize}
				\item Legislature's constraint holds $\forall \ T$ (but is tighter for large $T$)
				\item Lobby's constraint also holds $\forall \ T$ (but is tighter for small $T$)
					\begin{itemize}
						\item $\frac{\de - \de^{T+1}}{1-\de}$ is increasing in $T$, which means it gets smaller as you move toward the end of the punishment (there are fewer periods of punishment payoffs left)
					\end{itemize}
		\end{itemize}
\end{itemize}
	
\vskip.5in
\subsection{Conditions for punishment}
\begin{itemize}
	\item Legislature's condition (for class 4 above):
		\[
		  \frac{1 - \de^{T}}{1-\de} W(\ga(e^N),\tau^N) + \de^{T} W(\ga(e_a),\tau^a) \geq W(\ga(e^N),\cdot) + \frac{\de - \de^{T+1}}{1-\de}W(\ga(e^N),\tau^N) + \de^{T+1} W(\ga(e_a),\tau^a) 
		\]
			\begin{itemize}
				\item The $\cdot$ on the right hand side is where we fill in the best deviation. Given that this deviation can't affect the future path of play (any deviation triggers the resetting of the punishment phase), this should be a unilateral best response.
				\item The unilateral best response is $\tau^N$, so this ``deviation'' could not even be detected as a deviation. There is no incentive to deviate from the punishment given the strategy for punishing deviations from the punishment
				\item This same logic holds at any point during the punishment (replace $T$ with $r < T$). Logic stays the same.
			\end{itemize}
		\item Lobby's condition (for class 5 above)
			\[
			  \frac{1 - \de^{T}}{1-\de} \left[\pi(\tau^N) - e^N \right] + \frac{\de^{T}}{1-\de}\left[\pi(\tau^a) - e_a \right] \geq \pi(\tau^D) -e^D + \frac{\de}{1-\de} \left[ \pi(\tau^a) - e_a \right]
			\]
			\begin{itemize}
				\item What is best deviation? Would lobby want to choose $e^D=0$ and get $\tau(\ga(0))$?
				\item Given that the lobby can't affect the future path and that the legislature will provide the unilateral tariff that is its best response to $e^D=0$, the lobby's best deviation payoff is the unilateral best choice, i.e. the trade war lobbying effort / tariff.
				\item If the lobby deviates to something lower, it both gets a lower tariff in return (less than what is optimal for it in the one-shot game, which is what it was doing before) and gets stuck with the trade agreement tariff ever after
				\item NOTE: EVERYTHING hinges on whether this is a valid punishment, this two-stage, dynamic punishment
			\end{itemize}
\end{itemize}

\newpage
\section{Giovanni's asymmetric punishment suggestion}
\begin{itemize}
	\item For now, even though it's not realistic, leave punisher at $\tau^{tw}$ so that previous analysis for symmetric case Nash reversion goes through
	\item Have cheater go to $\tau = 0$ in part (2) of equilibrium construction
		\begin{itemize}
			\item It's not clear if there is an air-tight restriction on $e_{tw}$. Leg's condition would be easy if $e_{tw} = 0$, so going to go with that
		\end{itemize}
	\item Before worrying about punishment for punishment, go back to TA structure
		\begin{itemize}
			\item For legislature:
				\begin{multline}
  \frac{\de_\text{ML} - \de_\text{ML}^{T+1}}{1-\de_\text{ML}} \left[W_\text{ML}(\ga(e_b),\bta) - W_{\text{ML}}(\ga(e_b),0,\tau^{*tw}) \right] \geq \\
	W_{\text{ML}}(\ga(e_b),\tau^b(e_b),\tau^{*a}) - W_{\text{ML}}(\ga(e_b),\bta)
				\end{multline}
				\begin{itemize}
					\item Want to show whether or not this punishment can support lower $\bta$
					\item Only thing that changed from main construction is $\tau^{tw}$ to 0 in second term
					\item For a given $\tau^a$,
						\[
						  \frac{\partial \ov{e}}{\partial \tau^p} = -\frac{\frac{\partial \Omega}{\partial \tau^p}}{\frac{\partial \Omega}{\partial \ov{e}}} = -\frac{-\frac{\de_\text{ML} - \de_\text{ML}^{T+1}}{1-\de_\text{ML}}\frac{\partial }{\partial \tau^p} W_{\text{ML}}(\ga(e_b),\tau^p,\tau^{*tw}) }{(-)} = \frac{\frac{\de_\text{ML} - \de_\text{ML}^{T+1}}{1-\de_\text{ML}}\frac{\partial }{\partial \tau^p} W_{\text{ML}}(\ga(e_b),\tau^p,\tau^{*tw}) }{(-)}
						\]
							\begin{itemize}
								\item $\ov{e}$ is always $\geq$ both $\tau^{tw}$ and 0, so reducing $\tau^p$ reduces this welfare term. Then overall, reducing $\tau^p$ raises $\ov{e}$
								\item Intuition: punishment gets worse, so have to pay more for a break for any given $\tau^a$
							\end{itemize}
				\end{itemize}
			\item Lobby no longer wants ``trade war'' (really, punishment period). Only thing they can want is break tariff for one period. Then punishment punishes lobby too
			  \begin{equation}
					0 \geq \left[\pi(\tau^b(\ov{e})) - \ov{e}\right] - \pi(\tau^a) + \frac{\de_\text{L} - \de_\text{L}^{T+1}}{1-\de_\text{L}} \left[\pi(\tau^{tw}) -e_{tw} - \pi(\tau^a) \right]
					\label{ine:lob2}
				\end{equation}
becomes
				\begin{equation}
					0 \geq \left[\pi(\tau^b(\ov{e}')) - \ov{e}'\right] - \pi(\tau^{a'}) + \frac{\de_\text{L} - \de_\text{L}^{T+1}}{1-\de_\text{L}} \left[\pi(0) - \pi(\tau^{a'}) \right]
					\label{ine:lob3}
				\end{equation}
				\begin{itemize}
					\item We know from above that starting at the $\bta$ that's optimal when the punishment it $\btn$, $\ov{e}$ goes up when punishment tariff drops to 0. Then net profits (first bracketed term) goes down. The reasoning is as follows: net profits are largest at $\ov{e} = e_{tw}$. When $\tau^p = \tau^{tw}$, we know $\ov{e} \geq e_{tw}$. When $\tau^p = 0$, $\ov{e}$ still $\geq e_{tw}$, but there is slack in the condition so that we could have a lower $\tau^a$ than the one that leads to $e_{tw}$. But this is talking about initial impact, and as long as $\ov{e} \geq e_{tw}$ and $e$ goes up, net profits must decrease.
					\item For now we're holding $\tau^a$ constant so the profit at $\tau^a$ term stays the same
					\item The last bracketed term goes from positive in Expression~\ref{ine:lob2} to negative in Expression~\ref{ine:lob3}
					\item So the whole right side of Expression~\ref{ine:lob3} becomes negative. There is thus slack that can be exploited to reduce $\bta$.
					\item Reducing $\bta$ will tighten up the condition via the last two terms. It's not clear what happens with net profits because the $\left(\bta,\ov{e}\right)$ correspondence is concave. It may give additional slack, or it may slow you down a little.
				\end{itemize}
			\item To truly compare, have to assume we're at an interior for both (although entirely possible that whichever is higher may be on boundary and other is interior)
				\begin{itemize}
					\item \textbf{If the lobby's constraint loosens, does it do so everywhere?}
					\item $\frac{\partial \ov{e}}{\partial \bta}$ \textbf{concave seems to complicate the analysis}
						\begin{itemize}
							\item I've verified that it remains concave for this punishment scheme: it looks like Lemmas 1 and 2 could reverse because the denominator could be positive, but this requires $f(\de)\left[\pi(\tau^a) - \pi(0)\right] > \pi(\tau^b) - \pi(\tau^a)$. But since $\pi(\tau^b) - \pi(\tau^a) > \pi(\tau^b) - e_b - \pi(\tau^a)$, we can put the two together and see that in this case the lobby's constraint is violated. So Lemmas 1 and 2 hold as for the $\tau^{tw}$ punishment.
						\end{itemize}
					\item If there is slack, can we for sure lower $\bta$?
				\end{itemize}	
		\end{itemize}
	\item Punishment
		\begin{itemize}
			\item Lobby doesn't want to deviate
			\item Legislature would want to go to $\tau^D(0)$.
				\begin{itemize}
					\item Punish with $e=0$, $\tau > \tau^D(0)$ (needs to be higher so that lobby likes it)
				\end{itemize}
			\item \textbf{need to make sure payoffs for all of these are lower than eqm payoff for punisher}
		\end{itemize}
\end{itemize}

	
\end{document}