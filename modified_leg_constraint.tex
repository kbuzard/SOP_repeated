\documentclass[12pt]{article}

\addtolength{\textwidth}{1.4in}
\addtolength{\oddsidemargin}{-.7in} %left margin
\addtolength{\evensidemargin}{-.7in}
\setlength{\textheight}{8.5in}
\setlength{\topmargin}{0.0in}
\setlength{\headsep}{0.0in}
\setlength{\headheight}{0.0in}
\setlength{\footskip}{.5in}
\renewcommand{\baselinestretch}{1.0}
\setlength{\parindent}{0pt}
\linespread{1.1}

\usepackage{amssymb, amsmath, amsthm, bm}
\usepackage{graphicx,csquotes,verbatim}
%\usepackage[backend=biber,block=space,style=authoryear]{biblatex}
%\setlength{\bibitemsep}{\baselineskip}
%\usepackage[american]{babel}
%dell laptop
%\addbibresource{C:/Users/Kristy/Dropbox/Research/xBibs/tradeagreements.bib}
%\addbibresource{C:/Users/Kristy/Documents/Dropbox/Research/xBibs/tradeagreements.bib}
%\renewcommand{\newunitpunct}{,}
%\renewbibmacro{in:}{}


\DeclareMathOperator*{\argmax}{arg\,max}
\usepackage{xcolor}
\hbadness=10000

\newcommand{\ve}{\varepsilon}
\newcommand{\ov}{\overline}
\newcommand{\un}{\underline}
\newcommand{\ta}{\theta}
\newcommand{\al}{\alpha}
\newcommand{\Ta}{\Theta}
\newcommand{\expect}{\mathbb{E}}
\newcommand{\Bt}{B(\bm{\tau^a})}
\newcommand{\bta}{\bm{\tau^a}}
\newcommand{\btn}{\bm{\tau^n}}
\newcommand{\btw}{\bm{\tau^{tw}}}
\newcommand{\ga}{\gamma}
\newcommand{\Ga}{\Gamma}
\newcommand{\de}{\delta}

\begin{document}
\begin{center}
Modified Legislative Constraint
\end{center}

\vskip.2in
Here I change the legislative constraint so that future political economy weights are evaluated according to anticipated lobbying effort, instead of the current period, or ``break'' lobbying effort as in the draft submitted to the JIE
\begin{itemize}
	\item This is an attempt to keep the legislative constraint from collapsing when acknowledging the endogenous impact of lobbying effort on the break tariff, which I had not fully incorporated in the previous drafts
\end{itemize}

\vskip.3in
We can write the executives' joint problem as
\begin{equation}
  \max_{\bta} \frac{\bm{W_\text{E}}(\bta)}{1-\de_\text{E}} \hskip.2in \text{subject to}
  \label{prob:max}
\end{equation}
\begin{multline}
  \frac{\de_\text{ML} - \de_\text{ML}^{T+1}}{1-\de_\text{ML}} \left[W_\text{ML}(\ga(0),\bta) - W_{\text{ML}}(\ga(e_{tw}),\btw) \right] \geq
	W_{\text{ML}}(\ga(e_b),\tau^b(e_b),\tau^{*a}) - W_{\text{ML}}(\ga(e_b),\bta)
  \label{ine:leg2}
\end{multline}
\begin{center}
and
\end{center}
\begin{equation}
  e_b \geq \pi(\tau^b(e_b)) - \pi(\tau^a) + \frac{\de_\text{L} - \de_\text{L}^{T+1}}{1-\de_\text{L}} \left[\pi(\tau^{tw}) -e_{tw} - \pi(\tau^a) \right]
  \label{ine:lob2}
\end{equation}

\vskip.3in
\begin{itemize}
	\item Lobby's condition can't hold unless $\ov{e} \geq e_{tw}$
		\begin{itemize}
			\item Intuition: lobby's net profit is maximized at $e_{tw}$: if you're using an effort level below this to reduce the net profit and make the lobbying constraint hold, the lobby will just increase effort up to $e_{tw}$. So have to force lobbying effort above lobby's optimal level
			\item Only holds at $\ov{e} = e_{tw}$ if net profit is exactly $\pi(\tau^a)$, in which case lobby is indifferent between trade war and trade agreement.
		\end{itemize}
	\item So, the question is: when does there exist a $\tau^a$ such that $\ov{e} \geq e_{tw}$ and the pair $\left(\tau^a,\ov{e}\right)$  satisfies the lobby's constraint?
		\begin{itemize}
			\item Intuitively, how do you need to set $\tau^a$ so that $\ov{e}$, which is derived from Expression~\ref{ine:leg2} at equality, implies that $\tau^b$ is enough larger than $\tau^{tw}$ so that Expression~\ref{ine:lob2} holds?
			\item Or, am I even thinking about $\ov{e}$ anymore?
				\begin{itemize}
					\item Yes, but it has to be driven by setting $\tau^a$ high enough (otherwise, lobby would choose lower effort level, and Expression \ref{ine:lob2} would fail
				\end{itemize}
	
		\item Take $\tau^b = \tau^{tw}$ (note that lobby's constraint is still probably a problem, but this is a benchmark). Then it must be that $e_b = e_{tw}$. 
  		\begin{multline}
				\frac{\de_\text{ML} - \de_\text{ML}^{T+1}}{1-\de_\text{ML}} \left[W_\text{ML}(\ga(0),\bta) - W_{\text{ML}}(\ga(e_{tw}),\btw) \right] \geq	W_{\text{ML}}(\ga(e_{tw}),\tau^{tw},\tau^{*a}) - W_{\text{ML}}(\ga(e_{tw}),\bta)
			\end{multline}
				\begin{itemize}
					\item Can we say anything about what $\tau^a$ must be?
					\item Not sure if there is some $\tau^a$ that makes this equal. The left hand side is positive and largest when $\tau^a = 0$ and becomes negative and smallest when $\tau^a = \tau^{tw}$. The right hand side varies from positive to zero. I haven't tried very hard, but don't see a way to sign the total expression at $\tau^a =0$
				\end{itemize}
		\end{itemize}
\end{itemize}
		
\vskip.3in
Lobby's condition will hold if $\pi(\tau^{tw}) -e_{tw} = \pi(\tau^a)$
\begin{itemize}
	\item But we'd like to be able to reduce $\tau^a$ below this level so that lobby actually cares to create a trade war
	\item This is possible only if $\ov{e} \geq e_{tw}$. So $\tau^b$ is at minimum $\tau^{tw}$
	\item Looking back at Expression~\ref{ine:leg2}, if $\tau^a$ is above optimal tariff at $e=0$, raising $\tau^a$ reduces the LHS
		\begin{itemize}
			\item At the same time, raising $\tau^a$ reduces the RHS
		\end{itemize}
	\item Lemma 1 is going to need some modifications: $\frac{\partial \ov{e}}{\partial \tau^a} > 0$ IF increase in $\ga(e_{tw})$ term outweighs potential decrease in $\ga(0)$ term
\end{itemize}
	
\vskip.3in
To be clear about how legislature's decision is made:
\begin{itemize}
	\item In break stage, if $e^b$ is such that Expression~\ref{ine:leg2} 
		\begin{itemize}
			\item holds, choose $\tau^a$ (that is, there are lots of $e^b$ that make the leg unilaterally prefer a tariff higher than $\tau^a$, but dynamic incentives lead the leg to stick with the trade agreement)
			\item does not hold, choose $\tau^b$ (if breaking the agreement, want to choose static unilateral optimal tariff)
		\end{itemize}
\end{itemize}
	
\end{document}